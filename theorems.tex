%!TEX root = qmechanics.tex

\makeatletter
\def\thmt@refnamewithcomma #1#2#3,#4,#5\@nil{%
  \@xa\def\csname\thmt@envname #1utorefname\endcsname{#3}%
  \ifcsname #2refname\endcsname
    \csname #2refname\expandafter\endcsname\expandafter{\thmt@envname}{#3}{#4}%
  \fi
}
\makeatother

%\declaretheorem[name=Theorem,Refname={Theorem,Theorems}]{theorem}
\declaretheorem[name=Conjecture,Refname={Conjecture,Conjectures}]{conjecture}
\declaretheorem[name=Lemma,Refname={Lemma,Lemmas}]{lemma}
\declaretheorem[name=Remark,style=remark,Refname={Remark,Remarks}]{remark}

% =============== definition =======================
\declaretheoremstyle[
	headfont=\normalfont\bfseries,
	notefont=\mdseries, notebraces={(}{)},
	bodyfont=\normalfont,
	postheadspace=0.5em,
	mdframed={
        default,
		skipabove		= 5pt,
		skipbelow		= \topsep,
		leftmargin		= 0,
		rightmargin		= 0,
		innerleftmargin	= 4pt,
		innerrightmargin= 4pt,
		%hidealllines	= true,
		backgroundcolor	= orange!2,
		linewidth		= 1,
		linecolor		= orange!20,
		roundcorner		= 2pt,
	}
]{definitionstyle}
\declaretheorem[
	style		= definitionstyle,
	name		= Definition,
	Refname		= {Definition, Definitions}
]{defn}

% ============= question =====================
\declaretheoremstyle[
	headfont=\normalfont\bfseries,
	notefont=\mdseries, notebraces={(}{)},
	bodyfont=\normalfont,
	postheadspace=0.5em,
	mdframed={
        default,
		skipabove		= 5pt,
		skipbelow		= \topsep,
		leftmargin		= 0,
		rightmargin		= 0,
		innerleftmargin	= 4pt,
		innerrightmargin= 4pt,
		%hidealllines	= true,
		backgroundcolor	= red!8,
		linewidth		= 1,
		linecolor		= red!20,
		roundcorner		= 2pt,
	}
]{questionstyle}
\declaretheorem[
	style		= questionstyle,
	name		= Question,
	Refname		= {Question, Questions}
]{question}

% =============== example =======================
\declaretheoremstyle[
	headfont=\normalfont\bfseries,
	notefont=\mdseries, notebraces={(}{)},
	bodyfont=\normalfont,
	postheadspace=0.5em,
	mdframed={
        default,
		skipabove		= 5pt,
		skipbelow		= \topsep,
		leftmargin		= 0,
		rightmargin		= 0,
		innerleftmargin	= 4pt,
		innerrightmargin= 4pt,
		%hidealllines	= true,
		backgroundcolor	= gray!5,
		linewidth		= 1,
		linecolor		= gray!20,
		roundcorner		= 2pt,
	}
]{examplestyle}
\declaretheorem[
	style=examplestyle,
	name=Example
]{example}

% =============== theorem =======================
\declaretheoremstyle[
	headfont=\normalfont\bfseries,
	notefont=\mdseries, notebraces={(}{)},
	bodyfont=\normalfont,
	postheadspace=0.5em,
	mdframed={
        default,
		skipabove		= 5pt,
		skipbelow		= \topsep,
		leftmargin		= 0,
		rightmargin		= 0,
		innerleftmargin	= 4pt,
		innerrightmargin= 4pt,
		%hidealllines	= true,
		backgroundcolor	= green!5,
		linewidth		= 1,
		linecolor		= green!20,
		roundcorner		= 2pt,
	}
]{theoremstyle}
\declaretheorem[
	style=theoremstyle,
	name=Theorem
]{thm}


\declaretheoremstyle[
	headfont=\normalfont\bfseries,
	notefont=\mdseries, notebraces={(}{)},
	bodyfont=\normalfont,
	postheadspace=0.5em,
	mdframed={
        default,
		skipabove		= 5pt,
		skipbelow		= \topsep,
		leftmargin		= 0,
		rightmargin		= 0,
		innerleftmargin	= 4pt,
		innerrightmargin= 4pt,
		%hidealllines	= true,
		backgroundcolor	= blue!3,
		linewidth		= 1,
		linecolor		= blue!10,
		roundcorner		= 2pt,
	}
]{propositionstyle}
\declaretheorem[
	style=propositionstyle,
	name=Proposition
]{prop}
