\documentclass[a4paper]{report}

% =============== bibliography  ========================
\usepackage[
    backend=biber,
    style=authoryear-comp,
    maxcitenames=1,
    url=true,
    doi=true,
    sorting=ynt
]{biblatex}
\DeclareNameAlias{author}{first-last}
% use the following to print URL only if no DOI is defined
\DeclareSourcemap{
  \maps[datatype=bibtex]{
    \map[overwrite]{
      \step[fieldsource=doi, final]
      \step[fieldset=url, null]
      % \step[fieldset=eprint, null]
    }  
  }
}
% ================ packages to use Italian characters ================
% Note that the utf8 encoding only works if the file is saved in UTF-8 encoding
\usepackage[utf8]{inputenc}
\usepackage{lmodern}

\usepackage{csquotes}
\usepackage[italian,english]{babel}

% \usepackage[T1]{fontenc}
% ================ general page layout ================
\usepackage{fontspec}
\usepackage{microtype}
\usepackage{fullpage}
% ================ mathematics and fonts ================
\usepackage{bm} % provides more bold symbols

% bbold provides \mathbb{1}, but ruins letters, and \mathbb{1} itself is pixelated.
% \usepackage{bbold}

\usepackage{dsfont} % provides more \mathbb \smallskipymbols
\usepackage{amsmath,amssymb,amsthm,thmtools}
\usepackage{mathtools}
\usepackage{cases}
\usepackage{calc}
\usepackage{mathrsfs} % provides the nice \matchsrc font
% \usepackage{soulutf8} % provides \st and \ul
\usepackage[normalem]{ulem} %provides \sout command for striking through text
% =========== HYPERREF AND CO. =============
\usepackage{nameref, hyperref, cleveref}
\crefname{appsec}{Appendix}{Appendices}
\hypersetup{
  colorlinks   = true, %Colours links instead of ugly boxes
  urlcolor     = green!80!black, %Colour for external hyperlinks
  linkcolor    = blue, %Colour of internal links
  citecolor    = red!80!black %Colour of citations
}

\usepackage{physics} % provides Dirac notation and lots of other things

\usepackage{float} % provides the option H for includegraphics

\usepackage{graphicx}

\usepackage[usenames,dvipsnames,table]{xcolor}
\usepackage[skins]{tcolorbox}
\tcbuselibrary{breakable}

\newtcolorbox[auto counter]{depthbox}[2][]{%
    enhanced, breakable, title=Box~\thetcbcounter: #2 #1%
}

\usepackage[font=small,justification=centering]{caption} % tweak fonts in labels
\usepackage{subcaption}
% \newenvironment{Figure}
%   {\par\medskip\noindent\minipage{\linewidth}}
%   {\endminipage\par\medskip}

% placeins provides \FloatBarrier command. Also, with the section option it automatically puts a \FloatBarrier before each section (same with subsection option ecc.)
\usepackage{placeins}
\usepackage{multicol}
\usepackage{multirow,tabularx,booktabs}
\setlength{\columnsep}{1cm}

\usepackage{parskip} % suppress indenting new paragraphs

\usepackage[printonlyused,withpage,nohyperlinks,smaller]{acronym}
%!TEX root = entanglement.tex
% ======= THE FOLLOWING PATCH IS NECESSARY TO MAKE cleveref AND acronym WORK FINE TOGETHER ================
\makeatletter
\newcommand*{\org@overidelabel}{}
\let\org@overridelabel\@verridelabel
\@ifpackagelater{acronym}{2015/03/21}{% v1.41
  \renewcommand*{\@verridelabel}[1]{%
    \@bsphack
    \protected@write\@auxout{}{\string\AC@undonewlabel{#1@cref}}%
    \org@overridelabel{#1}%
    \@esphack
  }%
}{% older versions
  \renewcommand*{\@verridelabel}[1]{%
    \@bsphack
    \protected@write\@auxout{}{\string\undonewlabel{#1@cref}}%
    \org@overridelabel{#1}%
    \@esphack
  }%
}
\makeatother


\usepackage{tikz}
\usetikzlibrary{positioning}
\usetikzlibrary{backgrounds}
\usetikzlibrary{shapes.geometric}

\graphicspath{{./figures/}}
\addbibresource{bibliography.bib}
% code from https://tex.stackexchange.com/a/27107/30481

\DeclareFieldFormat{citehyperref}{%
  \DeclareFieldAlias{bibhyperref}{noformat}% Avoid nested links
  \bibhyperref{#1}}

\DeclareFieldFormat{textcitehyperref}{%
  \DeclareFieldAlias{bibhyperref}{noformat}% Avoid nested links
  \bibhyperref{%
    #1%
    \ifbool{cbx:parens}
      {\bibcloseparen\global\boolfalse{cbx:parens}}
      {}}}

\savebibmacro{cite}
\savebibmacro{textcite}

\renewbibmacro*{cite}{%
  \printtext[citehyperref]{%
    \restorebibmacro{cite}%
    \usebibmacro{cite}}}

\renewbibmacro*{textcite}{%
  \ifboolexpr{
    ( not test {\iffieldundef{prenote}} and
      test {\ifnumequal{\value{citecount}}{1}} )
    or
    ( not test {\iffieldundef{postnote}} and
      test {\ifnumequal{\value{citecount}}{\value{citetotal}}} )
  }
    {\DeclareFieldAlias{textcitehyperref}{noformat}}
    {}%
  \printtext[textcitehyperref]{%
    \restorebibmacro{textcite}%
    \usebibmacro{textcite}}}

%=================== CUSTOM MACROS ==============================
\newcommand{\CC}{\mathbb{C}}
\newcommand{\EE}{\mathbb{E}}
\newcommand{\NN}{\mathbb{N}}
\newcommand{\ZZ}{\mathbb{Z}}
\newcommand{\PP}{\mathbb{P}}
\newcommand{\QQ}{\mathbb{Q}}
\newcommand{\RR}{\mathbb{R}}

\newcommand{\mxm}{{m \!\times\! m}}
\newcommand{\NxN}{{N \!\times\! N}}
\newcommand{\xxy}[2]{{#1 \! \times \! #2}}
\newcommand{\dimens}[2]{{#1$\!\times\!$#2}}
\newcommand{\on}[1]{\operatorname{#1}}
\newcommand{\bs}[1]{\boldsymbol{#1}}


% \newcommand{\F}{\mathcal{F}}
% \newcommand{\D}{\mathcal{D}}
% \renewcommand{\calS}{\mathcal{S}}
% \newcommand{\I}{\mathcal{I}}
% \renewcommand{\L}{\mathcal{L}}

\newcommand{\calA}{{\mathcal{A}}}
\newcommand{\calB}{{\mathcal{B}}}
\newcommand{\calC}{{\mathcal{C}}}
\newcommand{\calE}{{\mathcal{E}}}
\newcommand{\calF}{{\mathcal{F}}}
\newcommand{\calG}{{\mathcal{G}}}
\newcommand{\calH}{{\mathcal{H}}}
\newcommand{\calI}{{\mathcal{I}}}
\newcommand{\calY}{{\mathcal{Y}}}
\newcommand{\calL}{{\mathcal{L}}}
\newcommand{\calM}{{\mathcal{M}}}
\newcommand{\calO}{{\mathcal{O}}}
\newcommand{\calP}{{\mathcal{P}}}
\newcommand{\calS}{{\mathcal{S}}}
\newcommand{\calU}{{\mathcal{U}}}
\newcommand{\calV}{{\mathcal{V}}}
\newcommand{\calW}{{\mathcal{W}}}
\newcommand{\calX}{{\mathcal{X}}}

\newcommand{\rmC}{{\mathrm{C}}}
\newcommand{\rmD}{{\mathrm{D}}}
\newcommand{\rmL}{{\mathrm{L}}}
\newcommand{\rmT}{{\mathrm{T}}}

\newcommand{\uncertain}[1]{{\color{red}\bfseries (#1)}}
\newcommand{\nice}[1]{{\color{green!50!black} #1}}
\newcommand{\bad}[1]{{\color{red!60!black} #1}}

\newcommand{\heading}[1]{\textbf{\color{orange}\textit{(#1)}}}

\DeclareMathOperator{\Ker}{Ker}
\DeclareMathOperator{\Imag}{Imag}
\DeclareMathOperator{\Var}{Var}
\DeclareMathOperator{\Pos}{Pos}
\DeclareMathOperator{\Herm}{Herm}
\DeclareMathOperator{\Id}{Id}


% ========== EXTERNAL FILES ==============================
%!TEX root = qmechanics.tex

\makeatletter
\def\thmt@refnamewithcomma #1#2#3,#4,#5\@nil{%
  \@xa\def\csname\thmt@envname #1utorefname\endcsname{#3}%
  \ifcsname #2refname\endcsname
    \csname #2refname\expandafter\endcsname\expandafter{\thmt@envname}{#3}{#4}%
  \fi
}
\makeatother

%\declaretheorem[name=Theorem,Refname={Theorem,Theorems}]{theorem}
\declaretheorem[name=Conjecture,Refname={Conjecture,Conjectures}]{conjecture}
\declaretheorem[name=Lemma,Refname={Lemma,Lemmas}]{lemma}
\declaretheorem[name=Remark,style=remark,Refname={Remark,Remarks}]{remark}

% =============== definition =======================
\declaretheoremstyle[
	headfont=\normalfont\bfseries,
	notefont=\mdseries, notebraces={(}{)},
	bodyfont=\normalfont,
	postheadspace=0.5em,
	mdframed={
        default,
		skipabove		= \topsep,
		skipbelow		= \topsep,
		leftmargin		= 0,
		rightmargin		= 0,
		innerleftmargin	= 4pt,
		innerrightmargin= 4pt,
		%hidealllines	= true,
		backgroundcolor	= orange!2,
		linewidth		= 1,
		linecolor		= orange!20,
		roundcorner		= 2pt,
	}
]{definitionstyle}
\declaretheorem[
	style		= definitionstyle,
	name		= Definition,
	Refname		= {Definition, Definitions}
]{defn}

% ============= question =====================
\declaretheoremstyle[
	headfont=\normalfont\bfseries,
	notefont=\mdseries, notebraces={(}{)},
	bodyfont=\normalfont,
	postheadspace=0.5em,
	mdframed={
        default,
		skipabove		= \topsep,
		skipbelow		= \topsep,
		leftmargin		= 0,
		rightmargin		= 0,
		innerleftmargin	= 4pt,
		innerrightmargin= 4pt,
		%hidealllines	= true,
		backgroundcolor	= red!8,
		linewidth		= 1,
		linecolor		= red!20,
		roundcorner		= 2pt,
	}
]{questionstyle}
\declaretheorem[
	style		= questionstyle,
	name		= Question,
	Refname		= {Question, Questions}
]{question}

% =============== example =======================
\declaretheoremstyle[
	headfont=\normalfont\bfseries,
	notefont=\mdseries, notebraces={(}{)},
	bodyfont=\normalfont,
	postheadspace=0.5em,
	mdframed={
        default,
		skipabove		= \topsep,
		skipbelow		= \topsep,
		leftmargin		= 6,
		rightmargin		= 6,
		innerleftmargin	= 4pt,
		innerrightmargin= 4pt,
		%hidealllines	= true,
		backgroundcolor	= gray!5,
		linewidth		= 1,
		linecolor		= blue!20,
		roundcorner		= 2pt,
	}
]{examplestyle}
\declaretheorem[
	style=examplestyle,
	name=Example
]{example}

% =============== theorem =======================
\declaretheoremstyle[
	headfont=\normalfont\bfseries,
	notefont=\mdseries, notebraces={(}{)},
	bodyfont=\normalfont,
	postheadspace=0.5em,
	mdframed={
        default,
		skipabove		= 5pt,
		skipbelow		= \topsep,
		leftmargin		= 0,
		rightmargin		= 0,
		innerleftmargin	= 4pt,
		innerrightmargin= 4pt,
		%hidealllines	= true,
		backgroundcolor	= green!5,
		linewidth		= 1,
		linecolor		= green!20,
		roundcorner		= 2pt,
	}
]{theoremstyle}
\declaretheorem[
	style=theoremstyle,
	name=Theorem
]{thm}


\declaretheoremstyle[
	headfont=\normalfont\bfseries,
	notefont=\mdseries, notebraces={(}{)},
	bodyfont=\normalfont,
	postheadspace=0.5em,
	mdframed={
        default,
		skipabove		= \topsep,
		skipbelow		= \topsep,
		leftmargin		= 0,
		rightmargin		= 0,
		innerleftmargin	= 4pt,
		innerrightmargin= 4pt,
		%hidealllines	= true,
		backgroundcolor	= blue!3,
		linewidth		= 1,
		linecolor		= blue!10,
		roundcorner		= 2pt,
	}
]{propositionstyle}
\declaretheorem[
	style=propositionstyle,
	name=Proposition
]{prop}


\date{\today}

\begin{document}

\title{Quantum mechanics}
\author{Luca Innocenti}
\maketitle

\tableofcontents

\chapter{Schr\"{o}dinger equation and stuff}

\begin{prop}
	The Hamiltonian is by definition the infinitesimal generator of the propagator:
	\begin{equation}
		\frac{d}{dt}\mathcal U(t,t_0) = (-iH(t))\mathcal U(t,t_0).
	\end{equation}
\end{prop}

\begin{thm}[Schr\"{o}dinger-Robertson uncertainty relations]
	Let $A$ and $B$ be Hermitian operators, and let us denote their variances on some state $\ket\psi$ as
	\[ \sigma_A^2\equiv\Var[A]=\EE[(A-\EE[A])^2] = \expval{A^2}-\expval{A}^2, \]
	and similarly for $\sigma_B^2$.
	Then the following holds:
	\begin{equation}
		\sigma_A^2 \sigma_B^2 \ge
		\abs{\expval{\{A,B\}}-\frac{1}{2}\expval A\expval B}^2
		+ \abs{\frac{1}{2i} \expval{[A,B]}}^2,
	\end{equation}
	and thus, in particular,
	\begin{equation}
		\sigma_A\sigma_B\ge\frac{1}{2}\abs{\expval{[A,B]}}.
	\end{equation}
\end{thm}
\begin{proof}
	The main observation is that $\sigma_A^2\equiv\Var(A)=\|A-\EE[A]\|^2$.
\end{proof}

\begin{prop}
	Let $\ket n,\ket m$ be (time-dependent) eigenvectors of the time-dependent Hamiltonian $H$, with
	$H\ket n=E_n\ket n$ and $H\ket m=E_m\ket m$, $E_n\neq E_m$. Then,
	\[
		\mel{m}{\partial_t H}{n}=(E_n-E_m)\braket{m}{\dot n}
		+ \delta_{nm} \dot E_n.
	\]
	\label{prop:components_dotH}
\end{prop}

\begin{remark}
	If two eigenstates orthonormal $\ket m, \ket n$ are degenerate,~\cref{prop:components_dotH} says that $\mel{m}{\dot H}{n}=0$. While this is true, it should be noted that it only holds if the eigenstates are chosen to be \emph{differentiable} at all times. This is a nontrivial point when there is a degeneracy: consider for example the case $H(t)=t\sigma_x$. If we choose $\ket0, \ket1$ as initial eigenstates, we get a discontinuity because the eigenstates for $t>0$ are $\ket\pm$.
\end{remark}

\begin{prop}
	Let $\ket\psi=\ket{\psi(t)}$ be the state at time $t$, evolving with a time-dependent Hamiltonian $H=H(t)$, whose instantaneous eigenvalues and eigenvectors are $E_n=E_n(t)$ and $\ket n=\ket{n(t)}$, respectively.
	Then,
	\begin{equation}
		\frac{d}{dt}\braket{n}{\psi} = -i\mel{n}{H}{\psi} + \braket{\dot n}{\psi}.
	\end{equation}
	Using~\cref{prop:components_dotH} this translates into
	\begin{equation}
		\frac{d}{dt}\braket{n}{\psi} =
		- i \mel{n}{H}{\psi}
		+ \sum_k' \braket{k}{\psi} \frac{\dot H_{nk}}{E_k - E_n}
		+ \sum_k'' \braket{\dot n}{k}\braket{k}{\psi},
	\end{equation}
	where the primed sum is over all eigenstates $E_k$ with $E_k\neq E_n$, while the doubly primed sum is over all eigenstates $E_k$ with $E_k=E_n$.
	\label{prop:derivative_expectation_values}
\end{prop}

\begin{prop}
	For any instantaneous eigenstate $\ket n\equiv\ket{n(t)}$, $\braket{\dot n}{n}$ is purely imaginary.
	\label{prop:ndotn_is_purely_imag}
\end{prop}
\begin{proof}
	Follows immediately by differentiating $\braket{n(t)}{n(t)}=1$.
\end{proof}

\begin{prop}
	If $\ket{\psi(t_0)}=\ket{n(t_0)}$, then~\cref{prop:derivative_expectation_values} becomes
	\begin{align}
		\left. \frac{d}{dt}\braket{n}{\psi}\right\rvert_{t=t_0} &= 
		-i E_n(t_0) + \braket{\dot n(t_0)}{n(t_0)}, \\
		\left. \frac{d}{dt}\braket{k}{\psi}\right\rvert_{t=t_0} &= 
		\braket{\dot k(t_0)}{n(t_0)} = \frac{\dot H_{kn}(t_0)}{E_k(t_0)-E_n(t_0)},\quad E_k(t_0)\neq E_n(t_0).
	\end{align}
\end{prop}

\begin{prop}
	Let $H(t)$ be a time-dependent Hamiltonian, and
	$\ket{\Psi}\equiv \sum_n c_n(t)e^{i\theta_n(t)}\ket{\psi_n(t)}$ with
	$\theta_n(t)\equiv-\int_0^t d\tau\,E_n(\tau)$.
	Then, assuming $E_n\neq E_m$ for all $n,m$, the Schr\"{o}dinger equation for $\ket{\Psi(t)}$ is equivalent to the following equations for the coefficients:
	\begin{equation}
	\begin{aligned}
		\dot c_n &= -c_n \braket{\psi_n}{\dot n}
		- \sum_m c_m e^{i(\theta_m-\theta_n)} \braket{ n}{\dot m} \\
		&= -c_n \braket{ n}{\dot n}
		- \sum_m c_m e^{i(\theta_m-\theta_n)} \frac{\mel{n}{\partial_t H}{ m}}{E_m-E_n}.
	\end{aligned}
	\end{equation}
\end{prop}

\begin{question}
	What about the degenerate case? Also, more precise statements about precision of this approximation? What is the probability of excited states being occupied?
\end{question}

Integrability of time-dependent Hamiltonian~\autocite{sinitsyn2018integrable}.

\begin{prop}
	Consider a state $\ket{\psi_0}=\sum_n c_n \ket n$ evolving according to a time-independent Hamiltonian $H$, with $H\ket n=E_n \ket n$. Let us denote with $\ket{\psi_t}=e^{-iHt}\ket{\psi_0}$ the state evolved at time $t$, and with $S(t)=\braket{\psi_0}{\psi_t}$ the associated overlap.
	We then have $S(t)=\sum_n \abs{c_n}^2 e^{-iE_n t}=\expval{\exp(-iHt)}$, and
	\begin{equation}
		\arccos\abs{S(t)} = \Delta H\, t + \mathcal O(t^3).
	\end{equation}
\end{prop}

\chapter{States}

\section{States}
\begin{defn}[State]
	Let $\calH$ be a complex Hilbert space, $\calB(\calH)$ the set of bounded operators on $\calH$,
	and $\calB_1(\calH)\subset\calB(\calH)$ the set of \emph{trace-class} operators on $\calH$,
	and $\calS(\calH)\subset\calB_1(\calH)$ be the set of positive trace-class operators with \textbf{unit trace}.
	The elements $\rho\in\calS(\calH)$ are called \textbf{states}.
	In other words, a state $\rho$ is a bounded, trace-class operator such that
	$\Tr(\rho)=1$ and $\rho\ge0$.
\end{defn}

\begin{prop}
	The space of trace-class operators $\calB_1(\calH)$ is a two-sided $*$-ideal of $\calB(\calH)$, and the set of states $\calS(\calH)$ is a convex, bounded subset of $\calB_1(\calH)$.
\end{prop}

\begin{defn}
	A state $\rho$ is \emph{pure} if $\rho^2=\rho$.
\end{defn}

\begin{defn}[Separable states]
	Let $\rho$ be a density matrix for a bipartite system $\mathcal H^A \otimes \mathcal H^B$.
	We say that $\rho$ is a \emph{product state} if there exist states $\rho^A$ and $\rho^B$ such that
	\begin{equation}
		\rho = \rho^A \otimes \rho^B.
	\end{equation}
	The state is called \emph{separable} if it can be written as a convex sum of product states, that is, if there are weights $p_i\in\RR, \sum_i p_i=1$, and product states $\rho_i^A \otimes \rho_i^B$, such that
	\begin{equation}
		\rho = \sum_i p_i \rho_i^A \otimes \rho_i^B.
	\end{equation}
	Otherwise, the state is said to be \emph{entangled}.
\end{defn}
Note that here $\rho_i^A$ and $\rho_i^B$ need not be pure here.

\begin{defn}[Separable and entangled multipartite states]
	An $N$-partite state is said to be \emph{fully separable} if its density matrix $\rho$ can be written as a convex sum of product states:
	\begin{equation}
		\rho = \sum_k p_k \rho_k^{(1)} \otimes \rho_k^{(2)} \otimes \cdots \otimes \rho_k^{(N)},
		\quad \text{with} \quad
		\sum_k p_k = 1
		\quad \text{and}\quad
		p_k > 0.
	\end{equation}
	Otherwise, the state is \emph{entangled}.
	A state is said to be \emph{genuine multipartite entangled} if all the subsystems are entangled with each other, not only some of them.
	This means in particular that there must be no way to write the state as tensor product of two different states, acting on some pair of subsystems.
\end{defn}
Note that in the above the single density matrices $\rho_k^{(j)}$ are in general mixed states (that is, not necessarily pure).

\begin{prop}
	Given a complete basis of Hermitian, traceless, orthogonal, operators, with normalisation $\Tr(\Lambda_i\Lambda_j)=d\delta_{ij}$, any state $\rho$ can be represented in terms of its expectation values over this basis as
	\begin{equation}
		\rho = \frac{1}{d}\left[ I + \sum_k \Tr(\rho\Lambda_k)\Lambda_k \right].
	\end{equation}
	This is the so-called \emph{Bloch space} representation of $\rho$.
\end{prop}

\begin{prop}
	The set of states $\calS(\calH)$, with $\calH$ $d$-dimensional, is homeomorphic to the \emph{Bloch space} $B(\RR^{d^2-1})$, which is defined as the set of expectation values of all states $\rho$ over a complete set of traceless, orthogonal observables:
	\begin{equation}
		B(\RR^{d^2-1})\equiv\left\{
			(\tr(\rho\Lambda_k))_{k=1}^{d^2-1}:\,\,\rho\in\calS(\calH)
		\right\} \subset \RR^{d^2-1}.
	\end{equation}
\end{prop}

\begin{prop}[\cite{lidar2013quantum} and \href{https://physics.stackexchange.com/a/425101/58382}{physics.SE answer}]
	The \emph{Bloch space} $B(\RR^{d^2-1})$ can be written as
	\begin{equation}
		B(\RR^{d^2-1}) \equiv \left\{
			\bs b\in\RR^{d^2-1}\,:\,\,\|\bs b\|\le\frac{1}{\abs{m(\Lambda_{\hat{\bs b}})}}
		\right\},
	\end{equation}
	where $\hat{\bs b}\equiv\bs b/\|\bs b\|$, $\Lambda_{\hat{\bs b}}\equiv\sum_k \hat{\bs b}_k \Lambda_k$, and $m(\Lambda_{\hat{\bs b}})$ denotes the smallest eigenvalue of $\Lambda_{\hat{\bs b}}$.
\end{prop}

\begin{prop}[a]
	The set of linear operators on an Euclidean space $\calX$ is spanned by the set of density matrices on $\calX$:
	\begin{equation}
		\on{span}\{\rho_{a,b}\,:\,\,(a,b)\in\Sigma\times\Sigma\}
		= \calL(\CC^\Sigma).
	\end{equation}
\end{prop}

\begin{defn}[Positive and completely positive maps, \emph{\cite{horodecki1996separability}}]
	Let $\mathcal H_1$ and $\mathcal H_2$ be 2 finite dimensional Hilbert spaces,
	and let $\mathcal A_1$ and $\mathcal A_2$ denote the set of operators acting on $\mathcal H_1$ and $\mathcal H_2$, respectively.
	Recall that $\mathcal A_i$ constitutes a Hilbert space (also called Hilbert-Schmidt space) with scalar product $\langle A, B\rangle = \Tr B^\dagger A$.
	The space of linear maps between $\mathcal A_1$ and $\mathcal A_2$ is denoted with
	$\mathcal L(\mathcal A_1, \mathcal A_2)$.
	We say that a map $\Lambda \in \mathcal L(\mathcal A_1, \mathcal A_2)$ is \emph{positive}
	if it maps positive operators in $\mathcal A_1$ into positive operators in $\mathcal A_2$,
	that is,
	if $A \ge 0$ implies $\Lambda(A) \ge 0$.
	Finally, we say that a map $\Lambda \in \mathcal L(\mathcal A_1, \mathcal A_2)$ is \ac{CP} if the induced map
	\begin{equation}
		\Lambda_n = \Lambda \otimes \mathds1 \, : \,\,
		\mathcal A_1 \otimes \mathcal M_n \to \mathcal A_2 \otimes \mathcal M_n
	\end{equation}
	is positive for all $n$.
	Here $\mathcal M_n$ stands for the set of complex $n\times n$ matrices,
	and $\mathds 1$ is the identity map.
	\label{def:positive_maps}
\end{defn}

Thus, from the above definition, a tensor product of a completely positive map with the identity maps positive operators into positive operators.
It is remarkable that there are at all positive maps that do not possess this 
property.

\section{Entropy and stuff}
Holevo's theorem~(\href{Watrous notes, chapter 12}{https://cs.uwaterloo.ca/~watrous/LectureNotes/CS766.Fall2011/12.pdf}).

\section{Purification of states}

\begin{defn}
	Let $\rho$ be a state acting on $\calH_A$.
	A state $\tilde\rho$ acting on $\calH_A\otimes\calH_B$ is said to be an \emph{extension} of $\rho$ if $\tr_B\tilde\rho=\rho$.
	A \emph{pure} state $\ket\psi\in\calH_A\otimes\calH_B$ is said to be a \emph{purification} of $\rho$ if $\tr_B\ketbra\psi=\rho$.
\end{defn}

\begin{prop}
	Let $\ket\psi$ be a purification of $\rho$: $\tr_1\ketbra\psi=\rho$. Then there are unitaries $U, V$ such that
	\begin{equation}
		\ket\psi=(U\otimes \sqrt\rho\,V)\sum_k \ket{kk}.
	\end{equation}
\end{prop}
\begin{proof}
	If $\ket\psi$ is a purification of $\rho$, then $\tr_1\ketbra\psi=\rho$.
	The components of the vector $\ket\psi$ can be expressed using two indices as $\psi_{ij}$. This means that we think of $\psi$ as a matrix, and thus apply the \ac{SVD} to it:
	\begin{equation}
		\psi_{ij} = \sum_k \lambda_k u_{ik} v_{jk},\quad
		\lambda_k\in\RR,
	\end{equation}
	where $u_{ik}, v_{jk}$ are elements of the unitary matrices $U$ and $V$, respectively. In vector notation, this reads
	\begin{equation}
		\ket\psi=\sum_k\lambda_k\ket{u_k}\otimes\ket{v_k}.
	\end{equation}
	Note that this is nothing but the Schmidt decomposition of $\ket\psi$.
	Imposing $\braket\psi=1$ gives $\sum_k \lambda_k=1$.
	The condition of the partial trace gives
	\begin{equation}
		\sum_k \lambda_k^2 \ketbra{v_k}=\rho
		\Longrightarrow \rho\ket{v_k}=\lambda_k^2\ket{v_k}.
	\end{equation}
	The result is then obtained by noticing that $U\ket k=\ket{u_k}$ and $V\ket k=\ket{v_k}$. Note how this also shows us that while the choice of $U$ does not matter, $V$ has to be the unitary sending $\ket k$ into the eigenbase of $\rho$.
\end{proof}

\begin{prop}
	Let $\rho$ be a state. Then, $\ket\psi$ is a purification of $\rho$ if and only if
	$\ket\psi=\sum_k\sqrt{p_k}\ket{u_k}\otimes\ket{v_k}$,
	for some orthonormal basis $\{\ket{u_k}\}_k$, and some eigenbasis $\{\ket{v_k}\}_k$ of $\rho$ with $\rho\ket{v_k}=p_k\ket{v_k}$.
\end{prop}
Note that in the above the choice of $\ket{v_k}$ is unique if and only if $\rho$ is nondegenerate.

\section{Fidelity}

\begin{defn}
	The \emph{fidelity} between two states $\rho$ and $\sigma$ is the quantity:
	\begin{equation}
		\calF(\rho,\sigma) \equiv \tr(\sqrt{\rho^{1/2}\sigma\rho^{1/2}}).
	\end{equation}
	\label{def:fidelity}
\end{defn}

\begin{prop}
	\Cref{def:fidelity} is equivalent to $\calF(\rho,\sigma)=\tr\lvert\sqrt\rho\sqrt\sigma\rvert$, where $|A|\equiv\sqrt{A^\dagger A}$.
	\label{def:fidelity_with_abs}
\end{prop}
\begin{proof}
	$(\sqrt\rho\sqrt\sigma)^\dagger (\sqrt\rho\sqrt\sigma) = \sqrt\sigma\rho\sqrt\sigma$.
\end{proof}

\begin{prop}
	The fidelity is \emph{symmetric}: $\calF(\rho,\sigma)=\calF(\sigma,\rho)$.
\end{prop}
\begin{proof}
	This is more easily shown from~\cref{def:fidelity_with_abs}, and follows from the more general fact that for any matrix $A$, $\tr|A^\dagger|=\tr|A|$.
	To see this, consider the \ac{SVD} of $A$:
	\begin{equation}
		A = \sum_k |\lambda_k| \ketbra{v_k}{w_k},
		\label{eq:svd_decomposition_for_A}
	\end{equation}
	with $|\lambda_k|^2\in\RR$ the eigenvalues of $A^\dagger A$, and $\{\ket{v_k}\}_k,\{\ket{w_k}\}_k$ the eigenbases of $AA^\dagger$ and $A^\dagger A$, respectively.
	Then, $|A|\equiv\sqrt{A^\dagger A}$ is written as
	\begin{equation}
		|A| = \sum_k |\lambda_k| \ketbra{w_k}{w_k},
		\label{eq:svd_decomposition_with_wk}
	\end{equation}
	so that the trace gives $\tr|A| = \sum_k |\lambda_k|$.
	Note that if we had defined $|A|$ as $|A|=\sqrt{AA^\dagger}$, we would have had $\ket{v_k}$ instead of $\ket{w_k}$ in~\cref{eq:svd_decomposition_with_wk}, but taking the trace we would have arrived to the same final result.
	The expression of $A^\dagger$ is equal to~\cref{eq:svd_decomposition_for_A}, except for $v_k$ and $w_k$ being switched. Using similar reasoning as above, we thus conclude that $\tr|A^\dagger|=\tr|A|$.
\end{proof}

\begin{prop}
	If $\rho$ is pure, with $\rho=\ketbra\psi$, then
	$\calF(\rho,\sigma)=\mel{\psi}{\sigma}{\psi}^{1/2}$.
\end{prop}

\begin{prop}
	If both $\rho=\ketbra{\psi_\rho}$ and $\sigma=\ketbra{\psi_\sigma}$ are pure, then
	$\calF(\rho,\sigma)=\lvert\braket{\psi_\rho}{\psi_\sigma}\rvert$.
\end{prop}

\begin{prop}
	If $[\rho,\sigma]=0$, then $\rho$ and $\sigma$ share an eigenbasis. Denoting with $p_k, q_k$ their eigenvalues, we have $\calF(\rho,\sigma)=\sum_k\sqrt{p_k q_k}$.
	In other words, for commuting states, the fidelity is equal to the \emph{Bhattacharyya distance} between the corresponding probability distribution.
\end{prop}

\begin{thm}[Uhlmann's theorem]
	Let us denote with $\ket{\psi_\rho}$ a \emph{purification} of the state $\rho$. Then,
	\begin{equation}
		\calF(\rho,\sigma)=\max_{\ket{\psi_\rho},\ket{\psi_\sigma}}\lvert\braket{\psi_\rho}{\psi_\sigma}\rvert.
	\end{equation}
\end{thm}

\begin{prop}
	Let $\{E_k\}_k$ be a POVM. The fidelity can be written as
	\begin{equation}
		F(\rho,\sigma)=\min_{\{E_k\}_k} F(\bs p,\bs q),
	\end{equation}
	where the minimum is taken over all possible POVMs, and $\bs p,\bs q$ are the probability distributions given by $\rho,\sigma$ for a given POVM:
	\begin{equation}
		p_k=\tr(E_k \rho),\quad q_k=\tr(E_k \sigma).
	\end{equation}
\end{prop}

\section{Convexity and other stuff}
The natural structure of the set $H_N$ of Hermitian $N\times N$ matrices is that of a real Hilbert space of dimension $N^2$ with the scalar product
$(X, Y) \equiv \Tr(XY^\dagger) = \Tr(XY)$.
The set $\mathcal D$ of density matrices is a subspace of $H_N$ defined as
\begin{equation}
	\mathcal D = \mathcal D_N = \{ \rho \in H_N \,:\,\, \rho \ge 0, \,\Tr\rho=1 \}.
\end{equation}
A density matrix $\rho$ has a spectral representation in terms of a complete set of orthonormal eigenvectors $\ket i \in \mathbb C^N$ with eigenvalues $\lambda_i \ge 0$:
\begin{equation}
	\rho = \sum_{i=1}^N \lambda_i \ketbra{i}{i},
	\qquad
	\braket{i}{j} = \delta_{ij},
	\qquad
	\sum_{i=1}^N \ketbra{i}{i} = \mathds1.
\end{equation}
The spectral representation is one particular ensemble representation of $\rho$.
The \emph{rank} $\Imag\rho$ of a density matrix $\rho$ is the number of eigenvalues $\lambda_i > 0$.
The \emph{kernel} (or null space) of $\rho$ is the set $\Ker(\rho)$ of states $\ket\psi$ such that $\rho\ket\psi = 0$.
Defining the operators $P$ and $Q$ as
\begin{equation}
	P \equiv \sum_{\mathclap{i\,:\,\lambda_i > 0}}\,\, \ketbra{i}{i}
	\qquad\text{and}\qquad
	Q \equiv 1 - P = \sum_{\mathclap{i\,:\,\lambda_i=0}}\,\, \ketbra{i}{i},
\end{equation}
we can see that they are Hermitian and project orthogonally onto the two complementary orthogonal subspaces $\Imag\rho$ and $\Ker\rho$.

A general Hermitian matrix $A$ is said to be \emph{positive}, or positive semidefinite,
if all of its eigenvalues are non-negative.
When this is the case, we write that $A\ge0$.
An equivalent way to state the positivity condition is to say that
$\mel{\psi}{A}{\psi} \ge 0$ for all $\ket\psi\in \mathbb C^N$.
The above condition, together with the spectral representation of a state $\rho$,
directly imply that $\mel{\psi}{\rho}{\psi} = 0 \Longleftrightarrow \rho\ket\psi=0$.

The fact that the positivity conditions $\mel\psi\rho\psi$ are linear in $\rho$ implies that $\mathcal D$ is a convex set, so that if $\rho$ is a proper convex combination of $\rho_1, \rho_2\in \mathcal D$, it is itself in $\mathcal D$.
Moreover, if $\rho=p \rho_1 + (1-p)\rho_2$ with $0<p<1$, then
$\Ker\rho = \Ker\rho_1 \cap \Ker\rho_2$.

A compact (closed and bounded) convex set is determined by its extremal points,
those points that are not convex combinations of other points in the set.
The extremal points of $\mathcal D$ are the pure states of the form
$\rho=\ketbra\psi$ with $\ket\psi\in\mathbb C^N$ and $\braket\psi=1$.
Thus, the spectral representation is an expansion of $\rho$ as a convex combination of $m$ extremal points in $\mathcal D$, where $m$ is the rank of $\rho$.

There are several equivalent ways to formulate the extremality condition of $\rho \in\mathcal D$.
The state $\rho$ is extremal in $\mathcal D$ if and only if
\begin{itemize}
	\item $\rho = \ketbra\psi$ with $\braket\psi=1$.
	\item The exists no $A\neq 0$ with $\Tr A=0$ and $PAP=A$.
	\item $PAP=A$ with $A$ Hermitian implies that $A=\rho$.
	\item There exists no $\rho'\in\mathcal D$ with $\rho'\neq\rho$ and $\Imag\rho'\neq\Imag\rho$.
\end{itemize}

\section{Necessary and sufficient characterizations of entanglement}
\Cref{def:positive_maps} states that if $\Lambda$ is \ac{CP},
then for any state $\rho$ we have $(\Lambda \otimes \mathds1)\rho \ge 0$.
On the other hand, if $\Lambda$ is a positive, non-\ac{CP} operator, there may be states $\rho$ such that
$(\Lambda \otimes \mathds1)\rho < 0$.
But if $\rho$ is a product state, $\rho = \rho_1 \otimes \rho_2$, then
$(\Lambda \otimes \mathds1)\rho = (\Lambda \rho_1) \otimes \rho_2$ is always positive,
even if $\Lambda$ is not \ac{CP}.
This simple observation suggest that whether a state remains positive when a non-\ac{CP} map acts on it may tell us something about its entanglement structure.

A first important result is that any entangled state can be linearly separated from the set of all separable states.
This follows from basic theorems in convex analysis.
The proof invokes the existence of a separating hyperplane between the compact convex set of separable density matrices on $\mathcal H_1 \otimes \mathcal H_2$ and a point,
the entangled density matrix $\rho$,
that does not belong to it.
This separating hyperplane can be characterized by a vector (in this case, an operator $A$),
that is normal to it.
The hyperplane is then defined as the set of density matrices $\sigma$ such that
$\Tr(A\rho) = 0$.
\begin{thm}[\emph{\cite{horodecki1996separability}}]
	For any \uline{inseparable} state $\rho \in \mathcal A_1 \otimes \mathcal A_2$,
	there exists an Hermitian operator $A$ such that
	\begin{equation}
		\Tr(A\rho) < 0
		\quad \text{and} \quad
		\Tr(A\sigma) \ge 0
		\,\, \text{ for any separable }\sigma.
	\end{equation}
\end{thm}

\begin{remark}
	The choice of ``origin'' in the above theorem is clearly irrelevant.
	We could have stated the result with $\Tr(A\rho) < c$ and
	$\Tr(A\sigma) \ge c$ for any $c \in \mathbb R$.
\end{remark}

\begin{lemma}[\emph{\cite{horodecki1996separability}}]
	A state $\rho \in \mathcal A_1 \otimes \mathcal A_2$ is \uline{separable} iff
	$\Tr(A\rho) \ge 0$
	for any Hermitian operator $A$ such that
	$\Tr(A P\otimes Q) \ge 0$, where $P$ and $Q$ are projections acting on
	$\mathcal H_1$ and $\mathcal H_2$, respectively.
	\label{lemma:horodecki96_separable_iff_TrArhoPositive}
\end{lemma}

\begin{remark}
	If an operator $A$ satisfies the condition $\Tr\big( A(P \otimes Q) \big) \ge 0$
	for any projectors $P$ and $Q$ then it is Hermitian.
\end{remark}

\begin{remark}
	\Cref{lemma:horodecki96_separable_iff_TrArhoPositive} can be generilized for infinitely dimensional Hilbert spaces.
	Namely, the condition becomes $\Tr(A\rho) \ge 0$ for any bounded $A$ such that
	$\Tr\big( A(P\otimes Q) \big) \ge 0$ for any projectors $P$ and $Q$.
\end{remark}

\begin{thm}[\emph{\cite{horodecki1996separability}}]
	Let $\rho$ be a state in the Hilbert space $\mathcal H_1 \otimes \mathcal H_2$.
	Then $\rho$ is separable iff for any positive map
	$\Lambda : \mathcal A_2 \to \mathcal A_1$,
	the operator $(\mathds1 \otimes \Lambda)\rho$ is positive.
\end{thm}

In the special case of bipartite states of 2 qubits, or 1 qubit and 1 qutrit,
the partial transposition criterion completely characterizes entanglement:
\begin{thm}[\emph{\cite{horodecki1996separability}}]
	A state $\rho$ acting on $\mathbb C^2 \otimes \mathbb C^2$
	or $\mathbb C^2 \otimes \mathbb C^3$ is separable iff
	its partial transposition is a positive operator: $\rho^{T_1} \ge 0$.
\end{thm}

\section{General structure of multipartite entanglement}

Higher dimensional entanglement without correlations~\autocite{klobus2018higher}.
\begin{lemma}[\emph{\cite{dr2000three}}]
	Let $\ket\psi, \ket\phi \in \mathbb{C}^n \otimes \mathbb{C}^m$ be 2 bipartite vectors, and suppose that for some local operator $A$ we have
	$\ket\phi = A\otimes \mathds1_B \ket\psi$.
	Then, the ranks of the corresponding reduced density matrices satisfy
	$r(\rho_A^\psi) \ge r(\rho_A^\phi)$ and
	$r(\rho_B^\psi) \ge r(\rho_B^\phi)$.
\end{lemma}
\begin{proof}
	Consider the Schmidt decomposition of $\ket\psi$:
	\begin{equation}
		\ket\psi = \sum_{1=1}^{n_\psi} \sqrt{\lambda_i^\psi} \ket{i} \ket{i},
		\quad \text{with } \lambda_i^\psi > 0
		\quad\text{ and }\quad n_\psi \le \min(n,m),
	\end{equation}
	and write the operator $A$  as
	$A = \sum_{i=1}^n \ketbra{\mu_i}{i}$,
	where the vectors $\ket{\mu_i} \in \mathbb{C}^n$ do not need to be normalized nor linearly independent.
	The reduced density matrices can therefore be written as
	\begin{equation}
		\rho_A^\psi = 
		\sum_{i=1}^{n_\psi} \lambda_i^\psi \ketbra{i}{i},
		\qquad
		\rho_A^\phi = 
		\sum_{i=1}^{n_\psi} \lambda_i^\psi \ketbra{\mu_i}{\mu_i}.
	\end{equation}
	If follows that $r(\rho_A^\phi) \le r(\rho_A^\psi)$, because $\rho_A^\phi$ can be written as a convex sum of \emph{at most} $n_\psi$ terms,
	and $n_\psi$ is the rank of $\rho_A^\psi$.
	Moreover, for an bipartite vector $r(\rho_A) = r(\rho_B)$, so that the second inequality of the lemma follows.
\end{proof}
Using the above lemma, we can prove the more general statement:
\begin{lemma}[\cite{dr2000three}]
	If two vectors $\ket\psi, \ket\phi \in \calH_1 \otimes \cdots \otimes \calH_N$
	are connected by local operators as
	$\ket\phi = A_1 \otimes \cdots \otimes A_N \ket\psi$, then the local ranks satisfy
	$r(\rho^\psi_k) \ge r(\rho^\phi_k)$ for all $k=1,...,N$.
\end{lemma}
\begin{thm}
	Two pure states of a multipartite system are equivalent under \ac{SLOCC}
	if they are related by a local invertible operator.
\end{thm}

\section{Entanglement criteria}
\subsection{Entanglement witnesses}

\begin{defn}[Noise tolerance, \cite{tth2009practical}]
	Let $\mathcal W$ be a witness detecting the pure quantum state $\rho$.
	In a real experiment the realised state will be mixed with some noise, so that the actual state will be of the form
	\begin{equation}
		\rho_{\text{noisy}}(p) = (1-p) \rho + p \rho_{\text{noise}},
	\end{equation}
	for some modeling of the noise encoded in $\rho_{\text{noise}}$.
	The \emph{noise tolerance} of the witness $\mathcal W$ is then defined as the largest $p$ such that we still have $\Tr{\mathcal W \rho_{\text{noisy}}}$.
\end{defn}

\begin{thm}
	
\end{thm}

\begin{thm}[\cite{tth2005detection}]
	For biseparable quantum states $\rho$ of $N$ qubits,
	denoting with $D_N^{(N/2)}$ the $N$-qubit Dicke state with $N/2$ excitations,
	the following inequality holds:
	\begin{equation}
		\Tr(\rho \ketbra*{D_{N/2,N}}{D_{N/2,N}})
		\le C_{N/2, N},
		\quad\text{with}\quad
		C_{N/2,N} \equiv \frac{1}{2}\frac{N}{N-1}.
	\end{equation}
	If follows that a projective witness for $\ket{D_{N/2,N}}$ is
	\begin{equation}
		\mathcal W = \frac{1}{2} \frac{N}{N - 1} \mathds 1 -
			\ketbra*{D_{N/2,N}}{D_{N/2,N}}.
	\end{equation}
\end{thm}

\begin{thm}[\cite{hffner2005scalable}]
	A projective witness for $\ket{D_{1, N}}$ is
	\begin{equation}
		\mathcal W = \frac{1}{2} \frac{N}{N - 1} \mathds 1 -
			\ketbra*{D_{N/2,N}}{D_{N/2,N}}.
	\end{equation}
\end{thm}

\begin{thm}[\cite{bergmann2013entanglement}]
	A projective witness for $\ket{D_{k, N}}$, $1 < k < N/2$, is given by
	\begin{equation}
		\mathcal W = \frac{N - k}{N} \mathds 1 - \ketbra*{D_{k,N}}{D_{k,N}}.
	\end{equation}
\end{thm}

\subsection{Schmidt number}
For a bipartite pure state that we write in its Schmidt decomposition:
$\ket\psi = \sum_{i=1}^k \sqrt{\lambda_k} \ket{a_i} \otimes \ket{b_i}$,
the number $k$ is the \emph{Schmidt rank} of the pure state.
This number is equal to the tank of the reduced density matrix (regardless of which space is traced).
A necessary condition for a pure state to be convertible by \ac{LOCC} to another pure state is that the Schmidt rank of the first pure state is larger than or equal to the Schmidt rank of the second pure state.
A generalization of the Schmidt rank to density matrices is the so-called \emph{Schmidt number}~\parencite{terhal2000schmidt}:
\begin{defn}[Schmidt number, \cite{terhal2000schmidt}]
	Let $\rho$ be a bipartite state.
	We say that $\rho$ has Schmidt number $k$ if
	i) for any decomposition of $\rho$, $\{p_i \ge 0, \ket{\psi_i}\}$ with $\rho = \sum_i p_i \ketbra{\psi}$, at least one of the vectors $\ket{\psi_i}$ has Schmidt rank at least $k$,
	and ii) there exists a decomposition of $\rho$ with all vectors $\ket{\psi_i}$ having Schmidt rank \emph{at most} $k$.
\end{defn}
For a pure state, the Schmidt number is simply equal to the Schmidt rank.
This immediately follows from remembering that a pure state $\rho$ only admits a single decomposition, of the form $\rho=\ketbra{\psi}$, and this decomposition clearly satisfies both (i) and (ii) with Schmidt rank $k$.
\begin{lemma}
	The set of separable density matrices $S_1$ is equal to the set of density matrices with Schmidt number equal to 1.
	In other words, the separable states are all and only the states with unitary Schmidt number.
\end{lemma}

\begin{thm}[\cite{terhal2000schmidt}]
	The Schmidt number of a density matrix cannot increase under \ac{LOCC}.
\end{thm}

\begin{defn}[$k$-positive maps, \cite{terhal2000schmidt}]
	Let $S_k$ denote the set of density matrices on $\calH_n\otimes\calH_n$ that have Schmidt number $k$ or less.
	The set $S_k$ is a convex compact subset of the set of density matrices, denotes by $S$,
	and $S_{k-1} \subset S_k$.
	A linear Hermiticity-preserving map $\Lambda$ is $k$-positive if and only if
	$(\mathds1 \otimes \Lambda)\ketbra\psi \ge 0$ for all $\ketbra\psi \in S_k$.
\end{defn}

\begin{thm}[\cite{terhal2000schmidt}]
	A density matrix $\rho$ has Schmidt number at least $k+1$ if and only if there exists a $k$-positive linear map $\Lambda_k: \mathcal M_n(\mathbb C)\rightarrow \mathcal M_n(\mathbb C)$ such that 
	$(\mathds1\otimes \Lambda_k)\rho < 0$.
\end{thm}

\chapter{Channels}
% Let $\rho\in\rmD(\calX)$ be a state.
A \emph{map} acting on an operator $X\in\rmL(\calX)$ is an operator $\Phi\in\rmL(\rmL(\calX),\rmL(\calY))$. We denote the set of such maps with $\rmT(\calX,\calY)$.

A given $\Phi\in\rmT(\calX,\calY)$ is said to be
\begin{itemize}
	\item \emph{unital} if $\Phi(I_{\calX})=I_{\calY}$, where $I_{\calX},I_{\calY}$ are the identities in $\calX,\calY$, respectively;
	\item \emph{trace-preserving} if $\Tr(\Phi(X))=\Tr(X)$;
	\item \emph{Hermitian preserving} if $\Phi(X^\dagger)=\Phi(X)^\dagger$, for all $X\in\rmL(\calX)$;
	\item \emph{positive} if $\Phi(X)\ge0$ for all $X\ge0$;
	\item \emph{completely positive} if $(\Phi\otimes I_n)\ge0$ for any $n\ge0$.
\end{itemize}

A map is said to be CPTP if it is trace-preserving and completely positive. CPTP maps are also referred to as \emph{channels}. We denote the set of channels with $\rmC(\calX,\calY)$.


\section{Representations}

\subsection{Natural representation}

Any operator $X\in\rmL(\CC^n)$ can also be understood as a vector, being $\rmL(\CC^n)$ itself a vector space with $\dim\rmL(\CC^n)=n^2$. A standard way of doing this involves choosing a basis $\{\sigma_k\}_k\subset\rmL(\CC^n)$ and performing the decomposition $X=\sum_k c_k\sigma_k$. We can then identify $X$ with the corresponding vector of coefficients $\bs c\equiv (c_k)_k$.
One such basis is $\{E_{ij}\}_{i,j=1}^n\subset\rmL(\CC^n)$, where $E_{ij}\equiv \ketbra{i}{j}$.
The corresponding coefficients of $X$ in such basis are simply its matrix elements. We denote such set of coefficients with $\on{vec}(X)\in\CC^{n^2}$, which is thus defined as
% \begin{equation}
	$\on{vec}(X)_{ij} \equiv X_{ij} \equiv \mel{i}{X}{j}$.
% \end{equation}
More formally, if $X\in\rmL(\calX,\calY)$, then $\on{vec}(X)\in\calY\otimes\calX$.

A map $\Phi\in\rmT(\CC^n,\CC^m)$ can then be represented as the matrix sending $\on{vec}(X)$ to $\on{vec}(\Phi(X))$. We denote this operator with $K(\Phi)\in\rmL(\CC^m\otimes\CC^m,\CC^n\otimes\CC^n)$, and refer to $K(\Phi)$ as the \emph{natural representation} of $\Phi$.
We thus have
\begin{equation}
	K(\Phi)_{ij,k\ell}
	\equiv \mel{i,j}{K(\Phi)}{k,\ell}
	= \mel{i}{\Phi(E_{k\ell})}{j}.
\end{equation}
More formally, if $\Phi\in\rmT(\calX,\calY)$, then $K(\Phi)\in\rmL(\calY\otimes\calY,\calX\otimes\calX)$.

\subsection{Choi representation}

\subsection{Stinespring representation}

\subsection{Kraus representation}

\section{Notable examples}

\begin{itemize}
	\item Isometric channels: $\Phi(X)=UXU^\dagger$ for some isometry $U$.
	\item Replacement channels;
	\item Entanglement-breaking channels. Channels $\Phi\in\rmC(\CC^n,\calY)$ such that for any bipartite $\rho\in\rmD(\calX\otimes\calH)$, the state $(\Phi\otimes I_{\rmL(\calH)})(\rho)\in\rmD(\calY\otimes\calH)$ is separable. We know that $\Phi\in\mathrm{EB}$ \emph{\textbf{iff}} its Choi is separable.
	An alternative characterisation is that $\Phi\in\mathrm{EB}$ \emph{\textbf{iff}} it can be written as $\Phi(\rho)=\sum_a \langle \mu(a),\rho\rangle \rho_a$, for some POVM $\{\mu(a)\}_a$ and collection of states $\{\rho_a\}_a$.
	\item Extreme channels;
	\item Pinching channels: $\Phi(X) =\sum_a \Pi_a X\Pi_a$ with $\{\Pi_a\}$ a projective measurement. The completely dephasing channel is an example of a pinching channel;
	\item Mixed-unitary channels;
\end{itemize}

\section{Adjoint channel}

Given a map $\Phi\in\rmT(\calX,\calY)$, we can always define its \emph{adjoint} $\Phi^*$. This is the unique map $\Phi^*\in\rmT(\calY,\calX)$ such that $\langle \Phi^*(Y),X\rangle=\langle Y,\Phi(X)\rangle$ for all $X\in\rmL(\calX)$ and $Y\in\rmL(\calY)$.

The adjoint of a channel is also sometimes referred to as the \emph{Heisenberg-picture representation} of the channel. This originates from the observation that performing a measurement $\mu\in\Pos(\calY)$ on a state $\rho\in\rmD(\calX)$ after evolution through $\Phi\in\rmC(\calX,\calY)$, amounts to performing the measurement $\Phi^*(\mu)$ on $\rho$:
\begin{equation}
	\langle \mu,\Phi(\rho) \rangle = \langle \Phi^*(\mu),\rho\rangle.
\end{equation}

\section{Standard notation (some bullshit to remove)}

\begin{equation}
	[K(\Phi)\on{vec}(\rho)]_{12} = K(\Phi)_{12,34}\rho_{34}
\end{equation}

\begin{equation}
	K(\Phi)
	= K(\Phi)_{12,34} \ketbra{12}{34}
	= K(\Phi)_{12,34} \ketbra{13}\otimes\ketbra{24}
	\equiv K(\Phi)_{12,34} E_{13}\otimes E_{24}
\end{equation}

\begin{equation}
	K_{12,34} = \mel{12}{K(\Phi)}{34}
	= \mel{1}{\Phi(E_{34})}{2}
	\equiv \mel{1}{\Phi(\ketbra{3}{4})}{2}
\end{equation}

The Choi operator of $\Phi$ is $J(\Phi)\equiv (\Phi\otimes I)(\PP_m)$, where $\ket m\equiv\sum_k \ket{kk}$ is the (unnormalised) maximally entangled state.
As an operator, we thus have $J(\Phi)\in\rmL(??)$.
Explicitly,
\begin{equation}
	J(\Phi) = (\Phi\otimes I)(\ketbra{12}\otimes\ketbra{12})
	= \Phi(\ketbra{1}{2})\otimes \ketbra{1}{2}
\end{equation}
\begin{equation}
	J(\Phi)_{12,34}
	\equiv \mel{12}{J(\Phi)}{34}
	= \mel{1}{\Phi(\ketbra{2}{4})}{3}
	= \mel{13}{K(\Phi)}{24}
	\equiv K(\Phi)_{13,24}.
\end{equation}

\section{Interesting properties and facts}

\begin{itemize}
	\item $\rmC(\calX,\calY)$ is \emph{convex};
	\item A channel is invertible \emph{\textbf{iff}} it is a unitary channel;
\end{itemize}

\begin{thm}
	Suppose $\Phi,\Psi\in\rmC(\calX,\calY)$ are such that $\Phi\circ\Psi=\Id$. If their Kraus operators are $\{A_a\}_a$ and $\{B_b\}_b$, respectively, then $A_a B_b=\lambda_{ab} I$ for some $\lambda_{ab}\in\CC$.
\end{thm}
\begin{proof}
	See Preskill notes.
\end{proof}

\section{Up-down notation for maps and states}

Another way to write maps is writing ``input indices'' low and ``output indices'' as high. For example:
\begin{equation}
	K(\Phi)_{12}^{34}\equiv K(\Phi)_{1|2}^{3|4} \equiv \mel{12}{K(\Phi)}{34}.
\end{equation}
In this notation, indices belonging to the same space are ``separated'' by vertical bars.
Note that this is not a separation between different \emph{parties}, but between different copies of the same space.
This notation has advantages when dealing with multipartite states.

A similar divider between the two copies of the space is to be used for states:
$\rho \equiv \rho_{12}\ketbra{1}{2}$ becomes $\rho_{1|2}\ketbra{1}{2}$.
If $\rho$ is bipartite, we have
\begin{equation}
	\rho \equiv \rho_{12,34} \ketbra{12}{34} \equiv \rho_{13|24} E_{13}\otimes E_{24},
	\qquad
	\rho_{12,34} \equiv \rho_{13|24}.
\end{equation}


Given a bipartite $\rho$, we then write the action of $\Phi$ on it as
\begin{equation}
	\Phi(\rho)_{12|34} =
	K(\Phi)_{12|34}^{56|78} \rho_{56|78}
\end{equation}

\section{Norms, distances, etc}

\begin{defn}[Induced trace norm]
	Let $\Phi\in\rmT(\calX,\calY)$. Its \emph{induced trace norm} is
	\begin{equation}
		\|\Phi\|_1 \equiv \max_{X\in\rmL(\calX): \|X\|_1\le 1}\|\Phi(X)\|_1.
	\end{equation}
	In words, this is the trace norm of $\Phi(X)$ maximised over all possible operators $X$ with trace norm less than $1$. (from Watrous)
\end{defn}

\chapter{CV formalism}

\section{Basic formulas}

\begin{equation}
	(a^\dagger)^n \ket m 
	= \sqrt{\frac{(m+n)!}{m!}} \ket{m+n}.
\end{equation}

\begin{equation}
	a^n \ket m = \sqrt{\frac{m!}{(m-n)!}} \ket{m-n},\quad n\le m.
\end{equation}

\section{Displacement operators}

\begin{defn}[Displacement operators]
	$D(\alpha) \equiv \exp(\alpha a^\dagger-\bar\alpha a)$ for $\alpha\in\CC$.
\end{defn}

\begin{prop}
	\begin{equation}
		D(\alpha)
		= e^{-\lvert\alpha\rvert^2/2}
			\exp(\alpha a^\dagger)\exp(-\bar\alpha a).
	\end{equation}
\end{prop}

\begin{prop}
	Displacement operators acting on the vacuum state generate coherent states:
	\begin{equation}
		D(\alpha)\ket0 = e^{-|\alpha|^2/2}\exp(\alpha a^\dagger)\ket0
		= \ket\alpha
		\equiv e^{-|\alpha|^2/2} \sum_{k=0}^\infty
			\frac{\alpha^k}{\sqrt{k!}} \ket k.
	\end{equation}
\end{prop}

\begin{prop}
	\begin{equation}
		D(\alpha)\ket1
		= e^{-|\alpha|^2/2}
			\sum_{k=0}^\infty \frac{\alpha^{k-1}}{\sqrt{k!}}\left(
			k - |\alpha|^2
			\right) \ket k
		= e^{-|\alpha|^2/2}
			\sum_{k=0}^\infty \frac{\alpha^{k}}{\sqrt{k!}}
			\left( \frac{ k - |\alpha|^2 }{ \alpha } \right) \ket k
	\end{equation}
\end{prop}
\begin{proof}
	We have
	\begin{equation}
		D(\alpha) \ket 1
		= e^{-|\alpha|^2/2}
		  \exp(\alpha a^\dagger) (\ket1 - \bar\alpha \ket0)
		= e^{-|\alpha|^2/2}
		  \sum_{k=0}^\infty \frac{\alpha^k}{k!} \left[
		  	\sqrt{(k+1)!}\ket{k+1}
		  	- \bar\alpha \sqrt{k!}\ket k
		  \right]
	\end{equation}
	and therefore
	\begin{equation}
		D(\alpha)\ket1
		= e^{-|\alpha|^2/2} \left[
			-\bar\alpha \ket 0
			+ \sum_{k=1}^\infty \frac{\alpha^{k-1}}{\sqrt{k!}}\left(
			k - |\alpha|^2
			\right) \ket k
		\right],
	\end{equation}
	which can be can also be written as in the proposition.
\end{proof}

\begin{prop}
	\begin{equation}
		D(\alpha) \ket \ell
		= e^{-|\alpha|^2/2}
		\sum_{s=0}^\infty \frac{\alpha^{s-\ell}}{\sqrt{\ell!s!}}
		\left(
		\sum_{k=0}^{\min(s,\ell)}
		(-1)^{\ell-k}|\alpha|^{2(\ell-k)} k!\binom{\ell}{k}\binom{s}{k}
		\right)
		\ket{s}
	\end{equation}
\end{prop}

\begin{proof}
	\begin{equation}
	\begin{gathered}
		D(\alpha) \ket \ell
		= e^{-|\alpha|^2/2}
		\exp(\alpha a^\dagger) 
		\sum_{k=0}^\ell \frac{(-\bar\alpha)^k}{k!} a^k \ket\ell \\
		= e^{-|\alpha|^2/2}
		\exp(\alpha a^\dagger) 
		\sum_{k=0}^\ell \frac{(-\bar\alpha)^k}{k!} \sqrt{\frac{\ell!}{(\ell-k)!}} \ket{\ell-k}
		= e^{-|\alpha|^2/2}
		\exp(\alpha a^\dagger) 
		\sum_{k=0}^\ell
		\frac{(-\bar\alpha)^{\ell-k}}{(\ell-k)!} \sqrt{\frac{\ell!}{k!}} \ket{k} \\
		= e^{-|\alpha|^2/2}
		\sum_{k=0}^\ell\sum_{j=0}^\infty
		\frac{(-\bar\alpha)^{\ell-k} \alpha^j}{j!(\ell-k)!} \sqrt{\frac{\ell!(k+j)!}{k!^2}} \ket{k+j} \\
		= e^{-|\alpha|^2/2}
		\sum_{s=0}^\infty \left(
			\sum_{k=0}^{\min(s,\ell)}
			\frac{(-\bar\alpha)^{\ell-k} \alpha^{s-k}}{(s-k)!(\ell-k)!} \sqrt{\frac{\ell!s!}{k!^2}} 
		\right) \ket{s} \\
		= e^{-|\alpha|^2/2}
		\sum_{s=0}^\infty \frac{\alpha^{s-\ell}}{\sqrt{\ell!s!}}
		\left(
		\sum_{k=0}^{\min(s,\ell)}
		(-1)^{\ell-k}|\alpha|^{2(\ell-k)} k!\binom{\ell}{k}\binom{s}{k}
		\right)
		\ket{s} 
	\end{gathered}
	\end{equation}
\end{proof}

\begin{prop}
	\begin{equation}
		D(\alpha)\PP_0 D^\dagger(\alpha) = \ketbra\alpha \equiv \PP_\alpha.
	\end{equation}
	\begin{equation}
		D(\alpha)\PP_1 D^\dagger(\alpha)
		= ...
	\end{equation}
\end{prop}

\begin{prop} to check:
	\begin{equation}
		D(\alpha) a D^\dagger(\alpha) = a + \alpha
	\end{equation}
\end{prop}

\section{Thermalised states}

\begin{defn}
	Given a state $\rho$, we define the corresponding \emph{thermalised state with average occupation number} $\bar n$ as
	\begin{equation}
		\Phi_{\bar n}^{\rm th}(\rho)
		\equiv \int d\mu \frac{e^{-\mu/\bar n}}{\bar n}
		D(\sqrt\mu)\rho D^\dagger(\sqrt\mu).
	\end{equation}
\end{defn}

\begin{prop}
	\begin{equation}
		\Phi^{\rm th}_{\bar n}(\PP_0)
		= \sum_{j,k=0}^\infty
			\frac{((j+k)/2)!}{\sqrt{j!k!}}
			\frac{1}{\bar n} \left(\frac{\bar n}{\bar n+1}\right)^{1+(j+k)/2}
			\ketbra{j}{k}
	\end{equation}
\end{prop}
Therefore in particular
\begin{equation}
	\langle j|\Phi^{\rm th}_{\bar n}(\PP_0)|j\rangle
	= \sum_{j=0}^\infty \frac{\bar n^j}{(1+\bar n)^{1+j}}.
\end{equation}

\begin{prop}
	\begin{equation}
		\langle j\rvert \Phi^{\rm th}_{\bar n}(\PP_1) \lvert j\rangle
		= \sum_{j=0}^\infty
			\frac{ \bar n^j }{ (\bar n+1)^{j+1} }
			\frac{ (j+ \bar n^2) }{ \bar n(\bar n + 1) }.
	\end{equation}
\end{prop}

\begin{proof}
	Remember that
	\begin{equation}
		D(\sqrt\mu)\PP_1 D(\sqrt\mu)
		= e^{-\mu}\sum_{j,k=0}^\infty \frac{\mu^{(j+k)/2 -1}}{\sqrt{j!k!}}
		(j-\mu)(k-\mu)
		\ketbra{j}{k}.
	\end{equation}
	Therefore
	\begin{equation}
	\begin{gathered}
		\Phi^{\rm th}_{\bar n}(\PP_1)
		\equiv \sum_{j,k=0}^\infty \ketbra{j}{k}
		\int_0^\infty d\mu \,
		\frac{e^{-\mu/\bar n}}{\bar n} e^{-\mu}
		\frac{\mu^{(j+k)/2 -1}}{\sqrt{j!k!}}
		(j-\mu)(k-\mu) \\
		= \sum_{j,k=0}^\infty \ketbra{j}{k}
		\frac{C^{-(j+k)/2}}{\bar n \sqrt{j!k!}}
		\int_0^\infty d\mu \,
		e^{-x} x^{(j+k)/2-1}
		(j-x/C)(k-x/C),
	\end{gathered}
	\end{equation}
	where $C\equiv 1+1/\bar n$. Evaluating the integral we have
	\begin{equation}
	\begin{gathered}
		= \sum_{j,k=0}^\infty \ketbra{j}{k}
		\frac{C^{-(j+k)/2}}{\bar n \sqrt{j!k!}}
		\left[
			jk ((j+k)/2-1)!
			-\frac{(j+k)}{C} ((j+k)/2)!
			+\frac{1}{C^2} ((j+k)/2+1)!
		\right].
	\end{gathered}
	\end{equation}
	In particular for $j=k$ we have
	\begin{equation}
	\begin{gathered}
		\langle j| \Phi^{\rm th}_{\bar n}(\PP_1) \lvert j\rangle
		= \sum_{j=0}^\infty
		\frac{C^{-j-2}}{\bar n j!}
		\left[
			j^2 (j-1)! C^2
			- 2j j! C
			+ (j+1)!
		\right] \\
		= \sum_{j=0}^\infty \frac{1}{\bar n}
		\left(\frac{\bar n}{\bar n + 1}\right)^{j+2}
		\left[
			j \left(\frac{\bar n+1}{\bar n}\right)^2
			- 2j \left(\frac{\bar n+1}{\bar n}\right)
			+ j+1
		\right] \\
		= \sum_{j=0}^\infty \frac{1}{\bar n}
		\left(\frac{\bar n}{\bar n + 1}\right)^{j+2}
		\left[
			j\left(
				\frac{(\bar n+1)^2 - 2\bar n(\bar n+1) + \bar n^2}{\bar n^2}
			\right)
			+1
		\right] \\
		= \sum_{j=0}^\infty \frac{1}{\bar n}
		\left(\frac{\bar n}{\bar n + 1}\right)^{j+2}
		\left[
			\frac{j}{\bar n^2}
			+1
		\right]
		= \sum_{j=0}^\infty 
			\frac{ \bar n^{j-1}}{(\bar n+1)^{j+2}}
			(j+ \bar n^2)
		= \sum_{j=0}^\infty
			\frac{ \bar n^j }{ (\bar n+1)^{j+1} }
			\frac{ (j+ \bar n^2) }{ \bar n(\bar n + 1) }
	\end{gathered}
	\end{equation}
\end{proof}

\begin{prop}
	\begin{equation}
		\Phi^{\rm th}_{\bar n}(\ketbra{0}{1})
		= 
	\end{equation}
\end{prop}

\begin{proof}
	\begin{equation}
	\begin{gathered}
		\Phi^{\rm th}_{\bar n}(\ketbra{0}{1})
		= \int_0^\infty d\mu \frac{e^{-\mu/\bar n}}{\bar n}
		\sum_{j,k=0}^\infty \ketbra{j}{k}
		e^{-\mu} \frac{1}{\sqrt{j!k!}}
		\mu^{\frac{j+k}{2} - \frac12} (k-\mu) \\
		= \sum_{j,k=0}^\infty \ketbra{j}{k}
		\frac{C^{(-\frac{j+k}{2}-\frac32)}}{\bar n\sqrt{j!k!}}
		\int_0^\infty dx
			e^{-x} x^{(\frac{j+k}{2}-\frac12)}
			(Ck - x) \\
		= \sum_{j,k=0}^\infty \ketbra{j}{k}
		\frac{1}{\bar n\sqrt{j!k!}}
		\left(\frac{\bar n}{\bar n + 1}\right)^{(j+k+3)/2}
		\Gamma\left(\frac{j+k+1}{2}\right)
		\left(
			Ck - \frac{j+k+1}{2}
		\right) \\
		= \sum_{j,k=0}^\infty \ketbra{j}{k}
		\frac{1}{\bar n\sqrt{j!k!}}
		\left(\frac{\bar n}{\bar n + 1}\right)^{(j+k+3)/2}
		\Gamma\left(\frac{j+k+1}{2}\right)
		\left(
			Ck - \frac{j+k+1}{2}
		\right) \\
		= \sum_{j,k=0}^\infty \ketbra{j}{k}
		\frac{1}{\bar n\sqrt{j!k!}}
		\left(\frac{\bar n}{\bar n + 1}\right)^{(j+k+3)/2}
		\Gamma\left(\frac{j+k+1}{2}\right)
		\left(
			\frac{ \bar n(k- j) + 2k - \bar n }{ 2\bar n }
		\right).
	\end{gathered}
	\end{equation}
	Thus in particular
	\begin{equation}
	\begin{gathered}
		\langle j| \Phi^{\rm th}_{\bar n}(\ketbra{0}{1}) |j\rangle
		= \frac{1}{\bar n j!}
		\left(\frac{ \bar n }{ \bar n + 1 }\right)^{j + 3/2}
		(j+1/2)!
		\left( \frac{ 2j - \bar n }{ 2\bar n } \right) \\
		=
		\frac{ \bar n^{ j - 1/2 } }{ (\bar n + 1)^{ j + 3/2 } }
		\frac{ (j+1/2)! }{ j! }
		\left( \frac{ 2j - \bar n }{ 2 } \right)
	\end{gathered}
	\end{equation}
\end{proof}

\begin{prop}
	The state $\ket0+\ket1$ evolving through a beasplitter with transmittivity $t$ results, looking only at the transmitted output branch, in the state
	\begin{equation}
		\rho_t
		\equiv
		\frac{2-T}{2} \PP_0
		+ \frac{T}{2} \PP_1
		+ \frac12 (t \ketbra{1}{0} + \mathrm{h.c.}).
	\end{equation}
	Applying $\Phi^{\rm th}_{\bar n}$ to this state, and focusing on the Fock-state probabilities, gives
	\begin{equation}
	\begin{gathered}
		P_j^{\mathrm{th};\bar n}
		= \frac{\bar n^j}{(\bar n + 1)^{j+1}}
		\left(
			\frac{2-T}{2}
			+ \frac{T}{2} \frac{j+\bar n^2}{\bar n(\bar n + 1)}
			+ \frac{\Re(t)}{\sqrt{\bar n(\bar n + 1)}}
			\frac{(j+1/2)!}{j!}
			\left( \frac{ 2j - \bar n }{ 2 } \right)
		\right).
	\end{gathered}
	\end{equation}
\end{prop}

\chapter{Bell nonlocality}

Review on Bell nonlocality~\autocite{brunner2014bell}.
Kochen-Specker theorem~\autocite{kochen1967problem}.
Fine's theorem~\autocite{fine1982joint,fine1982hidden}.

Consider a standard Bell's inequalities scenario: two parties, $A$ and $B$, sharing a bipartite state $\ket\Psi$. $A$ and $B$ can perform measurements on their respective parts of $\ket\Psi$. We will denote with $A$ and $B$ the random variables corresponding to their outcomes, and with $X$ and $Y$ the random variables corresponding to their choices of measurements.
The most general type of correlations that can be observed is of the form $p(ab|xy)$.

We want to understand whether the correlations between the measurements performed by $A$ and $B$ are compatible with a hidden variable theory.
By \emph{hidden variable theory} we mean a theory which tries to explain the correlations $p(ab|xy)$ under the assumptions of \emph{locality} and \emph{realism}.
Mathematically, this is expressed in the following assumption over $p$:
\begin{equation}
    p(ab|xy) = \sum_\lambda p(\lambda) p(a|x\lambda)p(b|y\lambda).
    \label{eq:prob_lhv_decomposition}
\end{equation}
The gist of Bell's theorem, in the CHSH form, is that~\cref{eq:prob_lhv_decomposition} is not compatible with \ac{QM}. In other words, the correlations produced by quantum mechanics cannot be explained via a probability $p(ab|xy)$ which satisfies~\cref{eq:prob_lhv_decomposition}.
The reason is that if~\cref{eq:prob_lhv_decomposition} is verified, then the outcome probabilities are a convex combination of \emph{factorised} outcome probabilities,
meaning that the expectation value of factorized functions $f(A,B)=f_1(A)f_2(B)$ over the outcomes gives
\begin{equation}
    \expval{f(A,B)}_{x,y} =
    \sum_\lambda p(\lambda) \expval{f(A,B)}_{x,y,\lambda} =
    \sum_\lambda p(\lambda) \expval{f_1(A)}_{x,\lambda} \expval{f_2(B)}_{y,\lambda}.
\end{equation}
A way to find a Bell inequality is then to find coefficients $\alpha_{x,y}\in\RR$ such that $\lvert \calS_\lambda\rvert\le C$ for all $\lambda$, with
\begin{equation}
    \calS_\lambda \equiv \Big\lvert
        \sum_{x,y} \alpha_{x,y} \expval{f(A,B)}_{x,y,\lambda}
    \Big\rvert.
\end{equation}
for this would directly imply that $\calS\equiv\sum_\lambda p_\lambda \calS_\lambda$ satisfies $|\calS|\le C$.


\begin{depthbox}[]{CHSH inequalities}
    Assume that $A$ and $B$ have two measurement choices, and two measaurement outcomes. We will conventionally denote with $x,y=\pm1$ and $a,b=\pm1$ the two possible measurement choices and outcomes, respectively.
    Let us define $E(x,y)$ as the expectation value of the product of the measurement outcomes measured by $A$ and $B$, when the measurement choices are $x$ and $y$:
    \begin{equation}
        E(x,y) \equiv \sum_{a,b} ab \, p(ab|xy).
    \end{equation}
    If the correlations satisfy~\cref{eq:prob_lhv_decomposition}, then we can expand this to
    \begin{equation}
    \begin{gathered}
        E(x,y) = \sum_\lambda p(\lambda) E_\lambda(x,y), \\
        E_\lambda(x,y) = \sum_a a\,p(a|x\lambda)\sum_b b\,p(b|y\lambda).
    \end{gathered}
    \end{equation}
    This means that $E(x,y)$ is really a convex combination of the expectation values $E_\lambda(x, y)$, each one of which is the expectation value of $AB$ \emph{for $A$ and $B$ uncorrelated with each other}:
    \begin{equation}
        E_\lambda(x,y) = \expval{A}_{x,\lambda} \expval{B}_{y,\lambda}.
    \end{equation}
    If $y,y'$ are two different measurement choices, then $E_\lambda(x,y)+E_\lambda(x,y')$ becomes
    \begin{equation}
        E_\lambda(x,y) + E_\lambda(x,y') =
        \sum_a a\,p(a|x\lambda) \sum_b b[p(b|y\lambda) + p(b|y'\lambda)].
    \end{equation}
    We will write this using the shorthand notation:
    \begin{equation}
        \expval{a_0 b_0}_\lambda + \expval{a_0 b_1}_\lambda =
        \expval{a_0}_\lambda (\expval{b_0}_\lambda + \expval{b_1}_\lambda).
    \end{equation}
    Let us now define the quantity $S_\lambda$, defined as
    \begin{equation}
    \begin{aligned}
        S_\lambda &\equiv
            E_\lambda(x,y) + E_\lambda(x,y') + E_\lambda(x',y) - E_\lambda(x',y') \\
        &= \expval{a_0}_\lambda (\expval{b_0}_\lambda + \expval{b_1}_\lambda)
        + \expval{a_1}_\lambda (\expval{b_0}_\lambda - \expval{b_1}_\lambda).
    \end{aligned}
    \end{equation}
    We then have
    \begin{equation}
        |S_\lambda| \le \lvert\expval{b_0} + \expval{b_1}\rvert
                      + \lvert\expval{b_0} - \expval{b_1}\rvert.
    \end{equation}
    Remembering that $\expval{a_i},\expval{b_i}$ are reals and lesser than $2$ in modulus, it easily follows that
    \begin{equation}
        \lvert S_\lambda\rvert \le 2.
    \end{equation}
    This is for example immediate in the case $\expval{a_i},\expval{b_i}\ge 0$.
\end{depthbox}

\section{Space of behaviors}
\newcommand{\NS}{\mathcal{N}\mathcal{S}}

Consider the set of all possible Bell correlation scenarios $\{p(ab|xy)\}_{abxy}$.
Let $\mathcal P$ denote the set of all such vectors.
If there are $m$ possible measurement choices, and $\Delta$ possible measurement outcomes, then $\mathcal P$ is the subspace of $\RR^{\Delta^2 m^2}$ containing all vectors $\mathbf p=\{p(ab|xy)\}_{abxy}$ such that
\begin{equation}
    p(ab|xy)\ge0\quad\text{ and }\quad \sum_{ab}p(ab|xy)=1.
\end{equation}
It follows that $\dim\mathcal P=(\Delta^2-1)m^2$.

A first limitation to $\mathcal P$ is to restrict to only the set of \emph{no-signaling} behaviors, which we will denote with $\NS$.
$\NS\subset\mathcal P$ is the set of behaviors whose marginals are independent of the other party's measurement settings:
\begin{equation}
    p(a|x)\equiv p(a|xy)\equiv\sum_b p(ab|xy),
\end{equation}
and similarly for $p(b|y)$.
It is shown in~\autocite{pironio2005lifting} that $\dim\NS=2(\Delta-1)m+(\Delta-1)^2m^2$.

A more restrictive constraint is represented by the set $\mathcal L$ of \emph{local behaviors}, defined as the element of $\mathcal P$ such that
\begin{equation}
    p(ab|xy)=\sum_\lambda q(\lambda) p(a|x,\lambda)p(b|y,\lambda).
\end{equation}
One can show that $\mathcal L\subset \NS\subset \mathcal P$.
\begin{prop}
    Nonlocal correlations $\mathbf p\notin \mathcal L$ can only exist if $\Delta\ge2$ \emph{and} $m\ge2$.
\end{prop}
\begin{proof}
    See~\href{https://physics.stackexchange.com/a/421622/58382}{physics.stackexchange.com/a/421622}.
    The $\Delta=1$ case is trivial. For the $m=1$ case, we want to prove that any probability distribution $p(a,b)$ can be written as
    \begin{equation}
        p(a,b)=\sum_\lambda q(\lambda)p(a|\lambda)p(b|\lambda).
    \end{equation}
    Consider a hidden variable $\lambda_{a,b}$ with $\Delta^2$ possible values, and a probability distribution $q(\lambda_{a,b})=p(a,b)$.
    Then we have the thesis if $p(a'|\lambda_{a,b})=\delta_{a,a'}$ and $p(b'|\lambda_{a,b})=\delta_{b,b'}$.
    Indeed, this proves not only that any $p(a,b)$ can be thought of as ``induced'' by a hidden variable, but that such hidden variable can always be chosen to \emph{deterministically} explain the probability distribution.
\end{proof}

\begin{prop}[\cite{fine1982hidden,brunner2014bell}]
    The set $\mathcal L$ of local correlations can always be expressed in terms of \emph{deterministic} local hidden-variable models.
\end{prop}
\begin{proof}
    \large{\color{red}Why?}
\end{proof}

\begin{prop}[\cite{brunner2014bell}]
    asd
\end{prop}


\chapter{Quantum resources}

Main sources: \cite{streltsov2016maximal}.

A number of different quantum features are considered as important resources for applications of quantum information theory.
Among these \emph{entanglement}, \emph{quantum discord}, and \emph{quantum coherence} have been identified as necessary ingredients for the successful implemntation of tasks such as quantum cryptography, quantum algorithms, and quantum metrology.
Quantum resources can be formally classified in the framework of resource theories,
where the state space is divided into free states and resource states.

\section{Resource theory of quantum coherence}
\newcommand{\setIncoherentStates}{\mathcal{I}}
Sources: \cite{aberg2006quantifying,baumgratz2014quantifying,winter2016operational,marvian2016how,streltsov2016quantum}.

The free states of this resource theory are called \emph{incoherent states},
that is, states which are diagonal in a fixed basis $\{\ket i\}$, like $\sigma = \sum_i p_i \ketbra{i}$.
Let us denote the set of such states with $\setIncoherentStates{}$.
The \emph{free operations} are the operations that cannot turn a free state into a resource state, so in this case the operations which cannot turn a mixed state into a coherent state.
There are several approaches to define free operations for mixed states:
\begin{enumerate}
	\item The historically first and most general approach, suggested in \cite{aberg2006quantifying}, is to define the set of \acp{MIO}: operations which cannot create coherence:
	\begin{equation}
	 	\Lambda_{\on{MIO}}(\sigma) \in \setIncoherentStates{}, \quad \forall \sigma \in \setIncoherentStates{}.
	 \end{equation}
	 \item Another important family is the set of \acp{IO}~\cite{baumgratz2014quantifying}.
	 These are operations which admit a Kraus decomposition $\Lambda_{\on{IO}}(\rho) = \sum_i K_i \rho K_i^\dagger$ with incoherent Kraus operators $K_i$, that is, $K_i \ket m \sim \ket n$.
	 \item See~\cite{streltsov2016maximal} for other definitions and extensions to multipartite systems.
\end{enumerate}
The \emph{amount of coherence} in a given state can be quantified via \emph{coherence monotones}.
These are nonnegative functions $\mathcal C$ which do not increase under the corresponding set of free operations.
A \ac{MIO} monotone is a nonnegative function $\mathcal C$ which does not increase under $\ac{MIO}$ operations:
$\mathcal C(\Lambda_{\on{MIO}}(\rho)) \le \mathcal C(\rho)$.
An important example are distance-based coherence monotones:
$\mathcal C(\rho) \equiv \inf_{\sigma\in\setIncoherentStates{}} D(\rho,\sigma)$,
with $D$ a suitable distance on the space of quantum states.
The most prominent example of distance-based measure is the \emph{relative entropy of coherence}
$\mathcal C_r(\rho) \equiv \min_{\sigma\in\setIncoherentStates{}} S(\rho \| \sigma)$
with the relative entropy
$S(\rho \| \sigma) \equiv \Tr(\rho \log_2\rho) - \Tr(\sigma\log_2\sigma)$.

\paragraph{Maximally coherent mixed states~\cite{streltsov2016maximal}}
Since coherence is a basis-dependent concept, a unitary operation will in general change the amount of coherence in a given state.
It is therefore natural to ask the question: \emph{which unitary maximizes the coherence of a given state $\rho$?}.
We therefore define the corresponding figure of merit
$\mathcal C_{\on{max}}(\rho) \equiv \max_U \mathcal C(U\rho U^\dagger)$.
While it is reasonable to believe that this maximization is out of reach, it turns out that the maximum can be evaluated in a large number of relevant scenarios, and that there exists a \emph{universal} maximally coherent mixed state, which does not depend on the particular choice of coherence monotone.
\begin{thm}[\cite{streltsov2016maximal}]
	For a fixed spectrum $\{ p_n \}$, the state
	\begin{equation}
		\rho_{\on{max}} = \sum_{n=1}^d p_n \ketbra{n_+},
	\end{equation}
	is a maximally coherent mixed state with respect to any \ac{MIO} monotone.
	Here $\{ \ket{n_+} \}$ denotes a \ac{MUB} with respect to the incoherent basis $\{ \ket{i} \}$,
	that is, $\lvert \braket{i}{n_+}\rvert^2 = 1/d$, where $d$ is the dimension of the Hilbert space.
\end{thm}
\begin{thm}[\cite{streltsov2016maximal}]
	For any distance-based coherence monotone with a contractive distance $D$,
	the following equality holds:
	$\mathcal C_{\on{max}} = \mathcal C(\rho_{\on{max}}) = D(\rho, \mathds{1} / d)$.
\end{thm}

\subsection{Coherence VS multipartite entanglement}
\parencite{regula2017converting,killoran2016converting}

In~\cite{regula2017converting} is presented a general formalism for the conversion of nonclassicality into multipartite entanglement.
They show that a faithful reversible transformation between the two resources is always possible, within a precise resource-theoretic framework.

Despite the common origin, entanglement and superposition can be formalized according to different resource theories:
the former being tied to the paradigm of spatially separated laboratories which can only implement local operations and classical communication for free,
while the second specified by the inability to create superpositions of the classical states for free.
Consequently, these two resources, like two currencies, enjoy different uses in quantum technologies.

\cite{regula2017converting} show that the presence of $k$-level nonclassicality in the state of a single $d$-level system is necessary and sufficient to create $(k+1)$-partite entanglement between the system and $k$ ancillas.
This builds on the result of \cite{killoran2016converting} that there always exists an isometry, consisting of adding an ancilla and applying a global unitary, which maps each pure state of nonclassical rank $k$ into a bipartite entangled pure state of Schmidt rank $k$.

\begin{thm}[\cite{regula2017converting}]
	\renewcommand{\DH}{\mathcal D(\calH{})}
	\newcommand{\calHanc}{\calH_{\on{anc}}}
	Let $\calH$ be a $d$-dimensional Hilbert space, $\DH{}$ the corresponding set of density operators,
	and $\calHanc{} \cong \calH{}$ the Hilbert space of an ancillary system.
	Then if the classical pure states $\{ \ket{\chi_i} \}_{i=1}^d$ form a linearly independent set spanning $\calH{}$,
	there exists an isometry $W: \calH{} \to \calH{} \otimes \calHanc{}$ such that for any state $\rho \in \DH{}$ we have
	$N_N(\rho) = N_S(W \rho W^\dagger)$.
\end{thm}
 
Following~\cite{ghne2005multipartite}, we define a pure state to be $k$-producible if it can be written as
$\ket\psi = \ket{\psi_1} \otimes ... \otimes \ket{\psi_m}$ with each $\ket{\psi_j}$ pertaining to at most $k$ parties,
and a mixed state $\rho$ to be $k$-producible if it can be written as a convex combination of $k$-producible pure states.
We call a state $\rho$ \emph{genuinely $k$-partite entangled} if it is $k$-producible but not $(k-1)$-producible.
Equivalently, under such conditions, we say that $\rho$ has \emph{entanglement depth} $D_E(\rho) = k$.
A 1-producible state $\rho$ has $D_E(\rho)=1$ and is fully separable.

\begin{thm}
	\renewcommand{\DH}{\mathcal D(\calH{})}
	\newcommand{\aHanc}{\mathcal H_{\on{anc}}}
	Let $\calH$ be a $d$-dimensional Hilbert space, and $\aHanc$ the Hilbert space of an ancillary system.
	Then if the classical pure states $\{ \ket{\chi_i} \}_{i=1}^d \in \calH$ form a linearly independent set spanning $\calH$, there exists an isometry $V: \calH\to\H\otimes\aHanc^{\otimes d}$
	such that for any state $\rho\in\DH$ with nonclassical number $N_N(\rho)=k$,
	$V \rho V^\dagger$ is genuinely $(k+1)$-partite entangled iff $\rho$ is nonclassical ($2\le k\le d$)
	and $V\rho V^\dagger$ is fully separable iff $\rho$ is classical ($k=1$).
\end{thm}

\chapter{Making sense of things}

\section{What does it mean to \emph{observe}?}

An \emph{observation} is the process of determining which one of a series of possible outcomes is realised. By definition, this entails a choice of \emph{measurement setting}, that is, a choice of interaction used to probe the unknown system.
Once a measurement has been chosen (say, \emph{look at the coin after flipping it}), the set of all possible states the system could be in is partitioned according to the possible outcomes. This is the distinction between \emph{macro} and \emph{micro} states.
But a system is not fully characterised by the way it behaves upon a given measurement: multiple different states might behave identically with respect to a fixed measurement. Therefore, knowledge of the outcomes in different measurement setups might be required for a full characterisation.
A question that follows naturally is then: \emph{are there relations between the outcomes in these possible choices of measurement?}
More specifically, we wonder whether the \emph{uncertainties} associated to such different measurements need to be somehow related.

QM teaches us that this is the case: there cannot be an arbitrary number of \emph{independent} measurements for a fixed system.
With ``\emph{independent}'' we mean here that the output probabilities corresponding to different measurement setups have zero mutual information, \emph{i.e.} knowing the outcome of one measurement gives us zero information about the outcome of another.
In reality, the situation in QM is even stronger: not only the measurement outcomes are independent, but they are also \emph{incompatible}, meaning that knowledge of an outcome in one measurement \emph{destroys} all knowledge we might have had regarding the outcomes in the other measurement.

From the QM formalism we know that such \emph{maximally incompatible} measurements produce probabilities satisfying $\sum_i p_i (1-p_i)\ge1/2$, where $i=1,2,3$ for a qubit system.
``Statistically unbiased'' means here that $p_i q_i=0$ implies full uncertainty with respect to the other measurement choices. The big question is: \textcolor{RedOrange}{\emph{\textbf{why not more than three unbiased measurements?}}}

Suppose knowing the outcome of $M_1$, we have full uncertainty on the outcome of $M_2$. This means the measurements correspond to ``incompatible quantities''.
One could try to understand this in terms of macro- and micro-states, but this approach misses the point. In particular, it makes the mistake of assuming that it makes sense to list faithfully the full set of states a system can be found in.
This is not the case when we consider situations in which the act of measurement itself perturbs the system.
In such scenario, it is not possible to make a list of all possible ``states'' of a system. Instead, it becomes necessary to identify the state by the way it behaves under several different measurement setups.

It must be stressed that such setup inherently assumes a separation between the system that is being probe and the rest of the universe.

\section{Decomplexifying QM}
A pure \emph{quantum state} $\psi$ is an element of a projective space $\mathbb{CP}^n\simeq S^{2n+1}/U(1)$.
Equivalently, it is an equivalence class of vectors in $\CC^{n}$ under the equivalence relation $\psi\sim\phi\Longleftrightarrow \psi=\lambda\phi$ for some $\lambda\in\CC$.
Using only real numbers, we can describe states as equivalence classes of vectors of $\psi\in\RR^{2n}\simeq \RR^n\otimes\RR^2$: $\psi\in\RR^{2n}\!/\!\sim$ where $\psi\sim\phi\Longleftrightarrow \psi=\alpha(I\otimes R)\phi$ for some $\alpha\in\RR$ and $R\in SO(2)$.
Equivalently, these are characterisable as orbits in the hypersphere $S^{2n-1}$.

In this notation, a linear map acting on a complex state, $A^c\psi^c$, translates into the action
\begin{equation}
	\psi\equiv \begin{pmatrix}
		\psi_R \\ \psi_I
	\end{pmatrix}
	\mapsto
	\begin{pmatrix}
		A_R \psi_R - A_I \psi_I \\
		A_R \psi_I + A_I \psi_R
	\end{pmatrix} \equiv A\psi \in \RR^n\otimes\RR^2.
\end{equation}
In other words, the linear transformation $A^c$ is represented as
\begin{equation}
	A^c \to A \equiv \begin{pmatrix}
		A_R & -A_I \\ A_I & A_R
	\end{pmatrix}.
\end{equation}
If $A^c$ is unitary, $(A^c)^\dagger A^c=A(A^c)^\dagger=I$, then $A_R^T A_R+A_I^T A_I = A_R A_R^T+A_I A_I^T = I$ and $A_R^T A_I = A_I^T A_R$.
From this we can see that $A^T A=AA^T=I_{2n}$ and $\det(A)=\lvert\det(A^c)\rvert^2=1$, that is, $A\in SO(2n)$.
Moreover, thus implies that $A_R$ and $A_I$ have the same left and right singular vectors, and with singular values summing to $1$:
\begin{equation}
	A_R = \sum_k \sqrt{s_k} \ketbra{u_k}{v_k},
	\qquad
	A_I = \sum_k \sqrt{1-s_k} \ketbra{u_k}{v_k}.
\end{equation}
Note the difference between these operators and unitary matrices: unitary matrices' singular values are all equal to $1$, and therefore unitaries send any orthonormal basis into another orthonormal basis. In contrast, here, different vectors $\ket{v_k}$ are treated differently, in that they correspond to different singular values.

One can also verify that $A^c$ unitary implies $A\in \on{Sp}(2n,\RR)$:
\begin{equation}\scalebox{0.9}{$\displaystyle
	A^T \begin{pmatrix}
		0 & I \\ -I & 0
	\end{pmatrix} A
	= \begin{pmatrix}
		A_R^T & A_I^T \\ -A_I^T & A_R^T
	\end{pmatrix}
	\begin{pmatrix}
		0 & I \\ -I & 0
	\end{pmatrix}
	\begin{pmatrix}
		A_R & -A_I \\ A_I & A_R
	\end{pmatrix} =
	\begin{pmatrix}
		-A_I^T A_R + A_R^T A_I & A_I^T A_I + A_R^T A_R \\
		-A_R^T A_R - A_I^T A_I & A_R^T A_I - A_I^T A_R
	\end{pmatrix}
$}\end{equation}

\subsection{Purely probabilistic route}

At a fundamental level, a \emph{qubit state} is something that always returns a binary answer when measured.
It is however crucial to observe that the outcome probabilities associated with a single measurement setup are not sufficient to fully characterise such a state: knowing that the probabilities in the $Z$ basis are $(0.5,0.5)$ does not tell us what the probabilities are in the $X$ basis.
Nonetheless, sometimes knowing the probabilities in one basis \emph{is} sufficient for a full characterisation. More specifically, knowing that the outcome is deterministic in a given basis, say $Z$, implies it having maximal entropy in \emph{mutually unbiased} bases.

For any state, there is always a measurement choice with a deterministic outcome, which alone would be sufficient to characterise the state. Can there be more than one? Sure, but that would probably amount to not be thinking about the state as a \emph{qubit} anymore.

Suppose there is a measurement $M$ which, performed on the state $\psi$, results in outcome probabilities $\calP(\psi,M)=(1,0)$.
There might be different a measurement $M'$ such that $\calP(\psi,M')=(0.5,0.5)$, and some other $M''$ such that $\calP(\psi,M'')=(3/4,1/4)$. Can we find a subset of measurements $\{M\}$ which allow us to optimally describe any possible outcome distribution resulting from a given state?
We would then also need to understand how changing $\psi$ would affect the results.

\begin{itemize}
	\item Such an optimal description would also come with a way to write measurements in terms of other measurements. What exactly would that correspond to, from a physical standpoint? Measuring $\alpha\sigma_x+\beta\sigma_y$ can be reduced to measuring $\sigma_z$ after evolving the state through an appropriate basis-changing unitary.
	This would correspond to saying that for any $M'$, there is some $\psi'$ such that $\calP(\psi,M)=\calP(\psi',M')$.
	Question is: is it necessary to discuss \emph{evolution} to only describe possible measurement outcomes?
	\item I guess the problem is to characterise what are the possible functional relationships $(\psi,M)\mapsto \calP(\psi,M)$. QM would have us think we need to parametrise $\psi$ as a complex vector (more specifically, $\psi\in\mathbb{CP}^n$) and $M$ as a Hermitian operator, and then $\calP(\psi,M)=\mel{\psi}{M}{\psi}$, but is this type of description really the only possible one?
	\item How can we inject into $M\mapsto \calP(\psi,M)$ the fact that there is some underlying ``thing'' (the state $\psi$) that is being measured in different ways? If $\psi$ is only characterised by $\calP(\psi,M_\psi)=(1,0)$ for some $M_\psi$, then there probably is no way to distinguish between different $M_\perp$ such that $\calP(\psi,M_\perp)=(0.5,0.5)$.
	At the same time, QM suggests that there \emph{cannot} be arbitrarily many such $M_\perp$. On the contrary, for a qubit, there are two and only two such ``unbiased'' measurements.
	Physically, that would mean that there cannot be more than two measurement setups leading to maximally entropic outcomes.
	There is probably some assumption on what kinds of ``measurements'' are accepted here, as one can imagine there being many ways to completely destroy the state in order to get unpredictable outcomes.
	\item One can imagine there being an underlying \emph{master distribution} which fully determines all outcomes (Bell be damned).
	Such distribution would have to also ``cause'' the choices of measurements, thus rendering the concepts of conditional probabilities somewhat non-fundamental.
	Note that we cannot have any ``hidden variable'' or similar parameter capable of predicting measurement outcomes. This is because we are not really describing the state \emph{per se}, but rather its interaction with its surroundings/measurement apparatus.
\end{itemize}

Suppose we have binary random variables $X_i$ which are all \emph{mutually unbiased}, meaning that $\Pr(X_i=x_i|X_j=x_j)=1/2$ for all $i,j$ and $x_i,x_j\in\{0,1\}$. I guess an important aspect here is that we are assuming that the system that is being measured is always the same. There must therefore be some bound on the total amount of information that can be stored in it.
At the same time, the random variables correspond to different measurements, \emph{i.e.} different ways in which we can interact with the system. Each one of these thus \emph{inject} some new information into the system, to some degree.
That is why we need to talk of a probability assignment of the form $\calP(\psi,M)$, which depends both on state \emph{and} measurement.

\section{What is this ``\emph{QM}'' you keep blabbering about?}

\section{Mach-Zender setup}

\heading{Summary of section}
In this section we revisit the standard Mach-Zender setup, but focusing on the relation between different probabilities.
Can everything be described exclusively in terms of probabilities?

\heading{The setting}
Consider the simple Mach-Zender setup, with a single photon passing through two BSs, with the inputs of the second equal to the inputs of the first one. We can write the corresponding evolution as
$a_1^\dagger\to a_1^\dagger + a_2^\dagger \to a_1^\dagger$, or equivalently $\ket0\to H\ket0\to H^2\ket0=\ket0$.

\heading{The paradox}
Measuring between the two BSs, the photon can be found in one of the two possible positions with equal probability. We can therefore meaningfully talk of the corresponding probability distribution $p(\on{left})=p(\on{right})=1/2$.
We can also talk about \emph{conditional} probabilities of observing the photon in, say, the first output of the second BS, when the photon has previously been (measured and) found in the $i$-th branch of the apparatus.
This will read $p(\text{out}_1|\text{left})=p(\text{out}_1|\text{right})=1/2$.
Classically, we should expect from these two facts that the final probability is given by
\begin{equation}
	p(\text{out}_1) =
	p(\text{out}_1|\text{left}) p(\text{left})
	+ p(\text{out}_1|\text{right}) p(\text{right}) = 1/2.
	\label{MZ:eq:conditional_probabilities_classical_law}
\end{equation}
This is \emph{not} what is observed, nor what QM predicts, which is $p(\text{out}_1)=1$.
Why is this so? What went wrong in the above argument?

\heading{The underlying wrong assumption}
The misguided assumptions in the above argument is the existence of a global probability distribution $p(x,y)$, with $x$ ``final position of the photon'' and $y$ ``position the photon is found in between the BSs''.
If there is such a probability distribution, then we can always write
\begin{equation}
	p(x) = p(x|Y=0) p(Y=0) + p(x|Y=1) p(Y=1).
\end{equation}
Note that the above is nothing but a rewriting of the very definition of marginal and conditional probabilities.
That this basic property of probability distributions fails in our MZ, means that it was wrong of us to assume the well-definedness of an underlying $p(x,y)$.
The existence of such an object, physically, means the possibility to, at least in principle, predict the value of $x$ and $y$.

\heading{``Measuring'' is itself an event to take into account}
The next natural question is ``why should $p(x,y)$ be ill-defined?''.
The answer lies in the observation that the \emph{act of measuring itself}, in this context, affects the possible outcomes of the apparatus. If no measurement is performed $p(\text{out}_1)=1$, but if a measurement is performed in between the BSs, we have $p(\text{out}_1)=1/2$ instead.
In other words, in~\cref{MZ:eq:conditional_probabilities_classical_law}, we did not account for the third possibility of no measurement performed between the BSs.
Doing that, we realise that the correct statement is quite trivial, because $p(\text{out}_1)=1$ only when no measurement is performed, which is an event incompatible with measuring the photon left or right (which are both sub-cases of the "measuring" instance).




\printbibliography

\section*{List of Acronyms}
\begin{acronym}
	\acro{QM}{Quantum Mechanics}
	\acro{CP}{Completely Positive}
	\acro{LOCC}{Local Operations and Classical Communication}
	\acro{SLOCC}{Stochastic Local Operations and Classical Communication}
    \acro{SLM}{Spatial Light Modulator}
    \acro{OAM}{Orbital Angular Momentum}
    \acro{HOM}{Hong-Ou-Mandel}
    \acro{MUB}{Mutually Unbiased Basis}
    \acro{MIO}{Maximally Incoherent Operations}
    \acro{IO}{Incoherent Operations}
    \acro{SVD}{Singular Value Decomposition}
\end{acronym}

\end{document}